\documentclass[a4paper]{article}
\usepackage{lrec2006}
\usepackage{url}
\usepackage{times}
\usepackage{latexsym}
\usepackage[small,bf]{caption}
\usepackage{xltxtra}

%\setlength\titlebox{6.5cm}    % You can expand the title box if you
% really have to

\title{Compiling Apertium morphological dictionaries with HFST and using them
in HFST applications}

%\author{Tommi A Pirinen, Francis M. Tyers}
%\address{University of Helsinki,
% Universitat d'Alacant\\
% FI-00014 University of Helsinki Finland,
% E-03071 Alacant Spain \\
%  \url{tommi.pirinen@helsinki.fi}, \url{ftyers@dlsi.ua.es}\\}

\date{\today}

\abstract{
  In this paper we aim to improve interoperability and re-usability of the
  morphological dictionaries of Apertium machine translation system by
  formulating a generic finite-state compilation formula that is implemented in
  in HFST finite-state system to compile Apertium dictionaries into general
  purpose finite-state automata. We demonstrate using of the resulting automaton
  in FST-based spell-checking system.\\
\Keywords{finite-state, dictionary, spell-checking}
}

\begin{document}

\maketitleabstract

\section{Introduction}

Finite-state automata are one of the most effective format for representing
natural language morphologies in computational format. The finite-state
automata, once compiled and optimised via process of minimisation are very
effective for parsing running text. This format is also used when running
morphological dictionaries in machine-translation system
Apertium~\cite{Apertium/2011}\footnote{\url{http://www.apertium.org}}. In this
paper we propose a generic compilation formula  to compile the
dictionaries into weighted finite-state automata with for use with any FST 
tool or application.  We implement this system using a free/libre
open source finite-state API
HFST~\cite{hfst/2011}\footnote{\url{http://hfst.sf.net}}. HFST is general
purpose programming interface utilising a selection of freely-available
finite-state libraries for the handling of finite-state automata.

While Apertium uses the dictionaries and the finite-state automata for machine
translation, HFST is used in multitude of other applications ranging from
basic morphological analysis~\cite{hfst/2011}
to end-user applications such as spell-checking~\cite{pirinen/2010/lrec} and
predictive text-entry for mobile phones~\cite{silfverberg/2011/cla}. In this
article we show how to generate automatically a spell-checker from an Apertium
dictionary and evaluate roughly the usability of the automatically generated
spell-checker.

The rest of the article is laid out as follows: In section \ref{sec:methods}
we describe the generic compilation formula for the HFST-based compilation of
Apertium dictionaries and the formula for induction of spell-checkers error
model from Apertium's dictionary. In section~\ref{sec:materials} we introduce
the Apertium dictionary repository and the specific dictionaries we use to
evaluate our systems. In section~\ref{sec:evaluation} we evaluate speed and
memory usage of compilation and application of our formula against Apertium's
own system and show that our system has roughly same coverage and explain
where the differences arise from.

\section{Methods}
\label{sec:methods}

The compilation of Apertium dictionaries is relatively straight-forward. We
assume here standard notations for finite-state algebra. The morphological
combinatorics of Apertium dictionaries are defined in following terms: There is
one set of root morphs (finite strings) and arbitrary number of named sets of
affix morphs called {\tt pardef}s. Each set of affix morphs is associated with a
name. Each morph can also be associated with a paradigm reference pointing to a
named subset of affixes. As an example, a language of singular and plural of
\emph{cat} and \emph{dog} in English would be described by root dictionary
consisting of morphs \texttt{cat} and \texttt{dog}, both of which point on the
right-hand side to pardef named \texttt{number}. The number affix morphs are
defined then as set of two morphs, namely \texttt{s} for plural marker and
empty string for singular marker.

Each morph can be compiled into single-path finite-state automaton\footnote{the
full formula allows any finite-state language as morph, compiled from regular
expressions, the extension to this is trivial but for readability we present
the formula for string morphs} containing the actual morph as string of UTF-8
arcs $m$. The morphs in the root dictionary are extended from left or right
sides by joiner markers iff they have a pardef definition there and each affix
dictionary is extended on the left (for suffixes) or right (for prefixes) by
the pardef name marker. In the example of \emph{cats, dogs} language this would
mean finite state paths \texttt{c a t NUMBER}, \texttt{d o g NUMBER},
\texttt{NUMBER s} and \texttt{NUMBER $\epsilon$}, where $\epsilon$ as usual
marks zero-length string\footnote{In the current implementation we have used
temporarily a special non-epsilon marker as this decreases the local
indeterminism and thus compilation time}.  These sets of roots and affixes can
be compiled into disjunction of such joiner delimited morphs.  Now, the
morphotactics can be defined as related to joiners by any such path that
contains joiners only as pairs of adjacent identical paradigm references, such
as \texttt{c a t NUMBER NUMBER s} or \texttt{d o g NUMBER NUMBER $\epsilon$},
but not \texttt{c a t NUMBER d o g NUMBER} or \texttt{NUMBER s NUMBER s}. The
finite-state formula for this morphotactics is defined by

\begin{equation}\label{formula:morphotax}
 M_x = (\Sigma \cup \bigcup_{x \in p} x x)^{\star},
\end{equation}

where $p$ is set of pardef names and $\Sigma$ the set
of symbols in morphs not including the set of pardef names.  Now the final
dictionary is simply composition of these morphotactic rules over the repetion
of affixes and roots:

\begin{equation}\label{formula:lexical}
(M_a \cup M_r)^{\star} \circ M_x,
\end{equation}

where $M_{a}$ is the disjunction of affixes with joiners, $M_{r}$ the
disjunction of roots with joiners, and $M_x$ the morphotactics defined in
formula~\ref{formula:morphotax}. This is a variation of morphology compilation
formula presented in various HFST documentation, such as~\cite{hfst/2011}.

\subsection{Implementation Details}

There are lot of finer details we will not thoroughly cover in this article, as
they are mainly engineering details. In this section we shortly summarise
specific features of HFST-based FST compilation that result in meaningful
differences in automaton structure or working. One of the main source of
differences is that HFST automata are two-sided and compiled only ones from the
source code whereas Apertium generates two different automata for analysis and
generation. In these automata the structure may be different, since Apertium
dictionaries have ways of marking morphs limited to generation or analysis
only, so they will only be included in one of the automatons. Our approach to
this is to use special symbols called flag-diacritics~\cite{beesley/2004} to
limit the paths as analysis only or generation only on runtime, but still
including all paths in the one transducer that gets compiled.

Another main difference in processing comes from the special word-initial, 
word-final and separate morphs that in Apertium are contained in separate
automata altogether, but HFST tools do not support use of multiple automata
for analysis, so these special morphs will be concatenated optionally to
beginning or end of the word, or disjuncted to the final automata respectively.
These special morphs include things like article \emph{l'} in French as bound
form.

\subsection{Creating a Spell-Checker Automatically}

To create a finite-state spell-checker we need two automata, one for the language
model, for which the dictionary compiled as described earlier will do, and one for the
error model~\cite{pirinen/2010/lrec}. A classic baseline error model is based
on the edit distance algorithm~\cite{levenshtein/1966,damerau/1964}, that
defines typing errors of four types: pressing extra key (insertion), not
pressing a key (deletion), pressing wrong key (change) and pressing two keys in
wrong order (swap). There have been many finite-state formulations of this, we
use the one defined in~\cite{schulz/2002,pirinen/2010/lrec}. The basic version
of this where the typing errors of each sort have equal likelihood for each
letters can be induced from the compiled language model, and this is what we
use in this paper. The induction of this model is relatively straightforward;
when compiling the automaton, save each unique UTF-8 codepoint found in the
morphs\footnote{The description format of Apertium requires declaration of
exemplar character set as well, but as this is only used in the tokenisation algorithm \cite{garrido-alenda02}
, which is not going to be used, we induce the set from the morphs}. For each 
character generate the identities in
start and end state to model correctly typed runs. For each of the error types
the generate one arc from initial state to the end state modelling that error,
except for swap which it requires one auxiliary state for each character pair.

\section{Materials}
\label{sec:materials}

The Apertium project hosts a large number of morphological dictionaries for
each of the languages translated. From these we have selected three
dictionaries to be tested: Basque from Basque-Spanish pair as it is 
released dictionary with the biggest on-disk size, Norwegian Nynorsk from the Norwegian pair as a language
that has some additional morphological complexity, such as compounding, and
Manx from  as a language that currently lacks spell-checking tools to
demonstrate the plausibility of automatic conversion of Apertium dictionary
into a spell-checker\footnote{We also provide a Makefile script to
recreate results of this article for any language in Apertium's repository}.

To evaluate the use of resulting morphological dictionaries and spell-checkers
we use following Wikipedia database 
dumps\footnote{\url{http://download.wikipedia.org/}}: 
\texttt{euwiki-20120219-pages-articles.xml.bz2}, 
\texttt{nnwiki-20120215-pages-articles.xml.bz2},
and \texttt{gvwiki-20120215-pages-articles.xml.bz2}. For the purpose of this
article we performed very crude cleanup and preprocessing to wikipedia data
picking up the text elements of the article and discarding most of wikipedia
markup naïvely\footnote{For details see the script in .}.

\section{Test Setting and Evaluation}
\label{sec:evaluation}

%To verify that the compilation works correctly we need to confirm that the
%resulting system can analyse and produce the same strings as original. For
%this, we perform two tests: one analysing running text and one generating all
%strings the dictionary contains (excepting cases that may produce
%infinite amount, such as regular expression morphs), and analysing those. The
%recall of our system is calculated as $\frac{\mathrm{correct}}{\mathrm{all}}$,
%where correct results means giving same set of answers as Apertium.
%The results of this test that shows how faithful the HFST recreation of the
%automaton is are given in table~\ref{table:recall}. Looking at the strings
%missing from HFST-version we can see following big classes of errors:
%foo (~\% of all errors) and bar (~\% of all errors), these are caused by
%features of underlying HFST system rather than the our finite-state formulation,
%and we expect them to be fixed in future versions of HFST.
%
%\begin{table}[h]
%\begin{center}
%\begin{tabular}{|l|r|r|}
%\hline
%\bf Language & \bf Running text & \bf Word-form list \\
%\hline
%Basque       & & \\
%Norwegian    & & \\
%Manx         & & \\
%\hline
%\end{tabular}
%\caption{Recall of HFST-based system against original
%\label{table:recall}}
%\end{center}
%\end{table}

To get one view on differences made by generic compilation formula instead of
direct automata building used by apertium we look at the created automata, this
will also give us rough idea of what its efficiency might be. In
table~\ref{table:graph-size} we give the counts of nodes and edges, in that
order, in the graphs compiled from the dictionaries. Note, that in case
of Apertium it is the sum of all the separate automata states and edges that
is counted. The small differences in sizes of graphs are mostly caused by
the different handling of generation vs. analysis mode. The difference in sizes
of automata on disk in is shown in table~\ref{table:size}.
The size of HFST automata can be attributed to the clever compression
algorithm used by HFST~\cite{silfverberg/2009/fsmnlp}.

\begin{table}[h]
\begin{center}
\begin{tabular}{|l|r|r|r|}
\hline
\bf Language & \bf Apertium lr & \bf Apertium rl & \bf HFST \\
\hline
Basque       & 30,114  & 34,005  & 34,824  \\
             & 59,321  & 68,030  & 68,347  \\
Norwegian    & 56,226  & 55,722  & 56,871  \\
             & 138,217 & 132,475 & 139,259 \\
Manx         & 13,055  & 12,955  & 12,920  \\
             & 28,220  & 27,062  & 27,031  \\
\hline
\end{tabular}
\caption{Size of HFST-based system against original (count of nodes first, then
edges)
\label{table:graph-size}}
\end{center}
\end{table}

\begin{table}[h]
\begin{center}
\begin{tabular}{|l|r|r|r|}
\hline
\bf Language & \bf Apertium lr & \bf Apertium rl & \bf HFST \\
\hline
Basque       & 252 KiB & 289 KiB & 1,7 MiB \\
Norwegian    & 558 KiB & 535 KiB & 3,7 MiB \\
Manx         & 108 KiB & 110 KiB & 709 KiB \\
\hline
\end{tabular}
\caption{Size of HFST-based system against original (as B on disk)
\label{table:size}}
\end{center}
\end{table}

To test efficiency we measure times of running various tasks.  The times and
memory usage have been measured using GNU \texttt{time} utility and
\texttt{getrusage} system call's \texttt{ru\_utime} field, averaged over three
test runs. The tests were performed on quad-core Intel Xeon E5450 @ 3.00~GHz
with 64 GiB of RAM. 

First we measure speed of analysing a full corpus with the result automaton.
The speed is measured in the table~\ref{table:speed}, in
seconds to precision that was available in our system. Curiously the results
do not give direct advantage to either of the system but it seems to
depend on the language which system is a better choice for corpus analysis.

\begin{table}[h]
\begin{center}
\begin{tabular}{|l|r|r|}
\hline
\bf Language & \bf Apertium & \bf HFST \\
\hline
Basque       & 32.0 s& 18.4 s \\
Norwegian    & 2.4 s & 5.5  s \\
Manx         & 1.6 s & 2.2  s \\
\hline
\end{tabular}
\caption{Speed of HFST-based system against original in corpus analysis
 (as s in user time)
\label{table:speed}}
\end{center}
\end{table}

Similarly we measure the speed of current compilation process in 
table~\ref{table:compile-speed}. In here there's an obvious advantage to
manual building of the automaton (see \cite{rojas2005} for the precise algorithm
used) over the finite-state algebra method, as is
in line with earlier results for lexc building in~\cite{pirinen/2009/sfcm}.

\begin{table}[h]
\begin{center}
\begin{tabular}{|l|r|r|}
\hline
\bf Language & \bf Apertium time & \bf HFST time \\
\hline
Basque       & 35.7 s & 160.0  s \\
Norwegian    & 6.6 s  & 200.2 s \\
Manx         & 0.8 s  & 11.2  s \\
\hline
\end{tabular}
\caption{Speed of HFST-based system against original in compilation
 (as s in user time)
\label{table:compile-speed}}
\end{center}
\end{table}

Finally we evaluate the usability of dictionaries meant for machine translation
as spell-checkers by running the finite-state spell checkers we produced
automatically through a large corpus and show the measure both speed and
quality of the results. The errors were automatically generated to Wikipedia
text's correct words using simple algorithm that may generate one Levenshtein
error per each character position at probability of $\frac{1}{33}$.  This test
shows only rudimentary results on the plausibility of using machine translation
dictionary for spell-checking; for more thorough evaluation of efficiency of
finite-state spell-checking see~\cite{hassan/2008}.

\begin{table}[h]
\begin{center}
\begin{tabular}{|l|r|r|}
\hline
\bf Language & \bf Speed (words/sec) \\
\hline
Basque       &  7,900 \\
Norwegian    &  9,200 \\
Manx         &  4,700 \\
\hline
\end{tabular}
\caption{Efficiency of spelling correction in artificial test setup, average 
  over three runs.}
\label{table:spelling}
\end{center}
\end{table}

\section{Future Work}
\label{sec:future}

In this article we showed a basic method to gain more inter-operability between
generic FST system of HFST and a specialised morphological dictionary writing
formalism of machine-translation system Apertium by implementing a generic
compilation formula to compile the language descriptions. In future research
we are leveraging this and other related formulas into automatic optimisation
of the final automata using the information present in the language description
to optimise instead of relying generic graph algorithms for the final minimised
result automata.
 
We demonstrated importing the compiled dictionary as a language model and
inducing error model for real-world spell-checking applications. Further
development in this direction should aim for interoperable formalisms, formats
and mechanisms for language models and end applications of all relevant
language technology tools.

\section{Conclusions}
\label{sec:conclusions}

In this article we have shown a general formula to compile morphological
dictionaries from machine-translation system Apertium in generic FST system of
HFST and using the result in HFST-based application of spell-checking.

%\section*{Acknowledgements}


\bibliographystyle{lrec2006}
\bibliography{lrec2011}


\end{document}
% vim: set spell:

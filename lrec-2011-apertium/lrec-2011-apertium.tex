\documentclass[10pt,a4paper]{article}
\usepackage{lrec2006}
\usepackage{url}
\usepackage{latexsym}
\usepackage{xltxtra}

%\setlength\titlebox{6.5cm}    % You can expand the title box if you
% really have to

\title{Compiling Apertium morphological dictionaries with HFST and using them
in HFST applications}

%\author{Tommi A Pirinen, Francis M. Tyers}
%\address{University of Helsinki,
% Universitat d'Alacant\\
% FI-00014 University of Helsinki Finland,
% E-03071 Alacant Spain \\
%  \url{tommi.pirinen@helsinki.fi}, \url{ftyers@dlsi.ua.es}\\}

\date{\today}

\abstract{
In this paper we aim to improve interoperability and re-usability of the
morphological dictionaries of Apertium machine translation system by
formulating a generic finite-state compilation formula that is implemented in
in HFST finite-state system to compile Apertium dictionaries into general
purpose finite-state automata. We demonstrate using of the resulting automaton
in FST-based spell-checking system.
\Keywords{finite-state, dictionary, spell-checking}
}

\begin{document}

\maketitleabstract

\section{Introduction}

Finite-state automata are one of the most effective format for presenting
natural language morphologies in computational format. The finite-state
automata, once compiled and optimised via process of minimisation are very
effective for parsing running text. This format is also used when running
morphological dictionaries in machine-translation system
Apertium~\cite{Apertium/2011}\footnote{\url{http://Apertium.sf.net}}. In this
paper we propose a generic compilation formula  to compile the
dictionaries into weighted finite-state automata with for use with any fst
tools and applications.  We implement this system using a free/libre
open source finite-state API
HFST~\cite{hfst/2011}\footnote{\url{http://hfst.sf.net}}. HFST is general
purpose programming interface utilising a selection of freely available
finite-state libraries for handling of finite-state automata.

While Apertium uses the dictionaries and the finite-state automata for machine
translation, HFST is used in multitude of other applications ranging from
basic morphological analysis~\cite{hfst/2011}
to end-user applications such as spell-checking~\cite{pirinen/2010/lrec} and
predictive text-entry for mobile phones~\cite{silfverberg/2011/cla}. In this
article we show how to generate automatically a spell-checker from an Apertium
dictionary and evaluate roughly the usability of the automatically generated
spell-checker.

The rest of the article is laid out as follows: In section \ref{sec:methods}
we describe the generic compilation formula for the HFST-based compilation of
Apertium dictionaries and the formula for induction of spell-checkers error
model from Apertium's dictionary. In section~\ref{sec:materials} we introduce
the Apertium dictionary repository and the specific dictionaries we use to
evaluate our systems. In section~\ref{sec:evaluation} we evaluate speed and
memory usage of compilation and application of our formula against Apertium's
own system and show that our system has roughly same coverage and explain
where the differences arise from.


\section{Methods}
\label{sec:methods}

The compilation of Apertium dictionaries is relatively straight-forward. We
assume here standard notations for finite-state algebra. The morphological
combinatorics of Apertium dictionaries are defined in following terms: There is
one set of root morphs (finite strings) and arbitrary number of named sets of
affix morphs called pardefs. Each set of affix morphs is associated with a
name. Each morph can also be associated with a paradigm reference pointing to a
named subset of affixes. As an example, a language of singular and plural of
\emph{cat} and \emph{dog} in English would be described by root dictionary
consisting of morphs \texttt{cat} and \texttt{dog}, both of which point on the
right-hand side to pardef named \texttt{number}. The number affix morphs are
defined then as set of two morphs, namely \texttt{s} for plural marker and
empty string for singular marker.

Each morph can be compiled into single-path finite-state automaton\footnote{the
full formula allows any finite-state language as morph, compiled from regular
expressions, the extension to this is trivial but for readability we present
the formula for string morphs} containing the actual morph as string of UTF-8
arcs $m$. The morphs in the root dictionary are extended from left or right
sides by joiner markers iff they have a pardef definition there and each affix
dictionary is extended on the left (for suffixes) or right (for prefixes) by
the pardef name marker. In the example of \emph{cats, dogs} language this would
mean finite state paths \texttt{c a t NUMBER}, \texttt{d o g NUMBER},
\texttt{NUMBER s} and \texttt{NUMBER $\epsilon$}, where $\epsilon$ as usual
marks zero-length string\footnote{in actual implementation we have used
temporarily a special non-epsilon marker since this decreases the local
indeterminism and thus compilation time}.  These sets of roots and affixes can
be compiled into disjunction of such joiner delimited morphs.  Now, the
morphotactics can be defined as related to joiners by any such path that
contains joiners only as pairs of adjacent identical paradigm references, such
as \texttt{c a t NUMBER NUMBER s} or \texttt{d o g NUMBER NUMBER $\epsilon$},
but not \texttt{c a t NUMBER d o g NUMBER} or \texttt{NUMBER s NUMBER s}. The
finite-state formula for this morphotax is defined by

\begin{equation}\label{formula:morphotax}
 M_x = (\Sigma \cup \bigcup_{x \in p} x x)^{\star},
\end{equation}

where p is set of pardef names and $\Sigma$ the set
of symbols in morphs not including the set of pardef names.  Now the final
dictionary is simply composition of these morphotactic rules over the repetion
of affixes and roots:

\begin{equation}\label{formula:lexical}
(M_a \cup M_r)^{\star} \circ M_x,
\end{equation}

where $M_{a}$ is the disjunction of affixes with joiners, $M_{r}$ the
disjunction of roots with joiners, and $M_x$ the morphotax defined in
formula~\ref{formula:morphotax}. This is a variation of morphology compilation
formula presented in various HFST documentation, such as~\cite{hfst/2011}.

To create a finite-state spell-checker we need two automata, a one for language
model, for which the dictionary compiled as described earlier will do, and one
error model~\cite{pirinen/2010/lrec}. A classical baseline error model is based
on the edit distance algorithm~\cite{levenshtein/1966,damerau/1964}, that
defines typing errors of four types: pressing extra key (insertion), not
pressing a key (deletion), pressing wrong key (change) and pressing two keys in
wrong order (swap). There have been many finite-state formulations of this, we
use the one defined in~\cite{schulz/2002,pirinen/2010/lrec}. The basic version
of this where the typing errors of each sort have equal likelihood for each
letters can be induced from the compiled language model, and this is what we
use in this paper. The induction of this model is relatively straightforward;
when compiling the automaton, save each unique UTF-8 codepoint found in the
morphs\footnote{the description format of apertium requires declaration of
exemplar character set as well, but to our sampling experience in many of the
actual descriptions this declaration does not match the reality so we opted to
induce the set from the morphs}. For each character generate the identities in
start and end state to model correctly typed runs. For each of the error types
the generate one arc from initial state to the end state modeling that error,
except for swap where it requires one auxiliary state for each character pair.

\section{Materials}
\label{sec:materials}

The Apertium project hosts a large number of morphological dictionaries for
each of the languages translated. From these we have selected three
dictionaries to be tested: Basque from Basque-Spanish pair as it is biggest
released dictionary, Norwegian Nynorsk from the Norwegian pair as a language
that has some additional morphological complexity, such as compounding, and
Manx from  as a language that currently lacks spell-checking tools to
demonstrate the plausibility of automatic conversion of Apertium dictionary
into a spell-checker\footnote{we also provide a Makefile script to
recreate results of this article for any language in Apertium's repository}.

To evaluate the use of resulting morphological dictionaries and spell-checkers
we use following Wikipedia database 
dumps\footnote{\url{http://download.wikipedia.org/}}: 
\texttt{euwiki-20120219-pages-articles.xml.bz2}, 
\texttt{nnwiki-20120215-pages-articles.xml.bz2},
and \texttt{gvwiki-20120215-pages-articles.xml.bz2}. For the purpose of this
article we performed very crude cleanup and preprocessing to wikipedia data
picking up the text elements of the article and discarding most of wikipedia
markup naively\footnote{for details see the script in .}.

\section{Test Setting and Evaluation}
\label{sec:evaluation}

To verify that the compilation works correctly we need to confirm that the
resulting system can analyse and produce the same strings as original. For
this, we perform two tests: one analysing running text and one generating all
strings the dictionary contains (excepting cases that may produce
infinite amount, such as regular expression morphs), and analysing those. The
recall of our system is calculated as $\frac{\mathrm{correct}}{\mathrm{all}}$,
where correct results means giving same set of answers as Apertium.
The results of this test that shows how faithful the HFST recreation of the
automaton is are given in table~\ref{table:recall}. Looking at the strings
missing from HFST-version we can see following big classes of errors:
foo (~\% of all errors) and bar (~\% of all errors), these are caused by
features of underlying HFST system rather than the our finite-state formulation,
and we expect them to be fixed in future versions of HFST.

\begin{table}[h]
\begin{center}
\begin{tabular}{|l|r|r|}
\hline
\bf Language & \bf Running text & \bf Word-form list \\
\hline
Basque       & & \\
Norwegian    & & \\
Manx         & & \\
\hline
\end{tabular}
\caption{Recall of HFST-based system against original
\label{table:recall}}
\end{center}
\end{table}

To test the efficiency we perform three tests: we compare the average times and
memory footprint of for analysing a big corpus with the final automata, we
compare the sizes and features of the resulting automata, and we compare the
average times for compilation of the dictionaries into automata. The times and
memory usage have been measured using GNU \texttt{time} utility and
\texttt{getrusage} system call's \texttt{ru\_utime} and 
\texttt{ru\_maxrss} fields, averaged over 5 test runs. The tests were performed
on quad-core Intel Xeon E5450 @ 3.00~GHz with 64 GiB of RAM. The speed is
measured in the table~\ref{table:speed}, in seconds to precision that was
available in our system. The memory consumption is in the
table~\ref{table:memory}, in MiB measured by maximum resident set size, this is
the largest amount of non-shared memory used by the program itself, not
including the memory of shared libraries loaded by the progran. 


\begin{table}[h]
\begin{center}
\begin{tabular}{|l|r|r|}
\hline
\bf Language & \bf Apertium time & \bf HFST time \\
\hline
Basque       & & \\
Norwegian    & & \\
Manx         & & \\
\hline
\end{tabular}
\caption{Speed of HFST-based system against original (as s in user time)
\label{table:speed}}
\end{center}
\end{table}

\begin{table}[h]
\begin{center}
\begin{tabular}{|l|r|r|}
\hline
\bf Language & \bf Apertium RSS & \bf HFST RSS \\
\hline
Basque       & & \\
Norwegian    & & \\
Manx         & & \\
\hline
\end{tabular}
\caption{Memory usage of HFST-based system against original (as MiB of RSS)
\label{table:memory}}
\end{center}
\end{table}

Finally we evaluate the usability of dictionaries meant for machine translation
as spell-checkers by running the finite-state spell checkers we produced
automatically through a large corpus and show the measure both speed and
quality of the results. The errors were automatically generated to Wikipedia
text's correct words using simple algorithm that may generate one Levenshtein
error per each character position at probability of $\frac{1}{33}$.  This test
shows only rudimentary results on the plausibility of using machine translation
dictionary for spell-checking; for more thorough evaluation of efficiency of
finite-state spell-checking see~\cite{hassan/2008}.

\begin{table}[h]
\begin{center}
\begin{tabular}{|l|r|r|}
\hline
\bf Language & \bf Speed & \bf Recall 1---$\infty$ \\
\hline
Basque       & & \\
Norwegian    & & \\
Manx         & & \\
\hline
\end{tabular}
\caption{Efficiency of spelling in artificial test setup (speed as s of user
time)
\label{table:spelling}}
\end{center}
\end{table}

\section{Future Work}
\label{sec:future}

In this article we showed a basic method to gain more inter-operability between
generic FST system of HFST and a specialised morphological dictionary writing
formalism of machine-translation system Apertium by implementing a generic
compilation formula to compile the language descriptions. In future research
we are leveraging this and other related formulas into automatic optimisation
of the final automata using the information present in the language description
to optimise instead of relying generic graph algorithms for the final minimised
result automata.
 
We demonstrated importing the compiled dictionary as a language model and
inducing error model for real-world spell-checking applications. Further
development in this direction should aim for interoperable formalisms, formats
and mechanisms for language models and end applications of all relevant
language technology tools.

\section{Conclusions}
\label{sec:conclusions}

In this article we have shown a general formula to compile morphological
dictionaries from machine-translation system Apertium in generic FST system of
HFST and using the result in HFST-based application of spell-checking.

%\section*{Acknowledgements}


\bibliographystyle{lrec2006}
\bibliography{lrec2011}


\end{document}
% vim: set spell:

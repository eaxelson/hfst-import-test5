\documentclass{llncs}
\usepackage{llncsdoc}
\usepackage[utf8x]{inputenc}
\usepackage{multirow}
\usepackage{caption}
\usepackage{url}
%
\begin{document}
%
\title{Using HFST for Creating Computational Linguistic Applications}
%
\author{Krister Lind\'{e}n \and Erik Axelson \and Senka Drobac \and\\
Sam Hardwick \and Tommi A Pirinen \and Miikka Silfverberg}

\institute{University of Helsinki\\
Department of Modern Languages\\
Unioninkatu 40 A\\
FI-00014 Helsingin yliopisto, Finland\\
\email{\{krister.linden, erik.axelson, senka.drobac, sam.hardwick,\\
tommi.pirinen, miikka silfverberg\}@helsinki.fi}}

\maketitle

\begin{abstract}
\sloppy HFST – Helsinki Finite-State Technology (\url{hfst.sf.net})
is a framework for compiling and applying linguistic descriptions
with finite-state
methods. HFST currently connects some of the most important finite-state
tools for creating morphologies and spellers into one open-source
platform and supports extending and improving the descriptions with
weights to accommodate the modeling of statistical information. HFST
offers a path from language descriptions to efficient language
applications.

HFST is designed for compiling morphologies and for that purpose it
contains open-source replicas of xfst, lexc and twolc which are
well-known and well-researched tools for morphology building. They
support both parallel and cascaded application of transducers. In
addition, HFST offers the capability to train and apply part-of-speech
taggers on top of the morphologies using weighted finite-state transducers.

With the morphology and tagger capabilities a number of applications
have been created, e.g. spellers for close to 100 languages and
hyphenators for approximately 40 different languages. The spellers were
derived from open-source dictionaries and integrated with OpenOffice and
LibreOffice. E.g. a full-fledged Greenlandic speller, which is a
polyagglutinating language, is currently available only via HFST for
OpenOffice. Using the tagger capability of HFST, we have created an
improved spelling suggestion mechanism for words in context as well as
better predictive text input for mobile phones for highly inflecting
languages like Finnish. Other writer’s tools created with HFST include
inflecting thesauri and translation dictionaries. We also offer lexicon
compilation and preprocessing for the Apertium machine translation
software.

For the processing of large corpora, e.g. for text indexing purposes, a
high-performing look-up facility is provided with HFST offering
110.000-440.000 words per second of morphological processing. A recent
extension of the look-up facility offers a utility to create named
entity recognizers for information extraction purposes on top of the
lookup. 
\keywords{no, keywords, yet}
\end{abstract}

\section*{Introduction}
% Krister

\section{Building Morphologies}

\subsection{Introduction}
% Miikka

\subsection{Parallel Rules with Negation and Regular Expression}
% Miikka

\subsection{Cascaded Rules: Explanations and Examples}
% Senka

\subsection{Morphosyntax and Morphological Formulae}

\subsubsection{For Lexicographers}
% Tommi

\subsubsection{For Grammarians}
% Tommi & Miikka

\subsection{Performance}
% Erik

\section{Building Taggers}

\subsection{Introduction}
% Miikka

\subsection{Including Morphologies without Harmonizing Tagsets}
% Miikka

\subsection{Optimization}
% Miikka

\section{Applying Transducers}

\subsection{Introduction}
% Tommi

\subsection{Spellers}

\subsubsection{Creating Spellers}
% Tommi

\subsubsection{Checking Strings and Generating Suggestions}
% Sam

\subsubsection{Ranking Suggestions}
% Tommi & Miikka

\subsection{Synonym and Translation Dictionaries}
% Krister

\section{Extending Transducers for Pattern Matching}
% Sam

\section{Discussion}

\section{Conclusion}

\subsubsection*{Acknowledgments}

\bibliographystyle{splncs03}

%\bibliography{hfst2012}

\end{document}
% vim: set spell

\documentclass[t,12pt,pdftex]{beamer}
\usepackage{helvet}
\usepackage{times}
\usepackage{courier}

\usepackage[T1]{fontenc}
\usepackage[english]{babel}

\usepackage{amssymb}
\usepackage{amsmath}
\usepackage{amsfonts}
\usepackage{graphicx}
\usepackage{color}
\usepackage{url}
\usepackage{textpos}
\usepackage{xspace}
\usepackage{array}

\graphicspath{{./fig/}}

% theme options: hy/ml/hum, rovio/sinetti, hiit
% default: hy,rovio

%\usetheme[hy]{HY}
%\usetheme[hy,sinetti]{HY}
\usetheme[hum,rovio]{HY}
%\usetheme[ml,rovio]{HY}
%\usetheme[ml,rovio,hiit]{HY}


\title{Building Real-World Finite-State Spell-Checkers With HFST\\
\scriptsize{in FSMNLP 2012 at Donostia}}
%[Tut: HFST Spellers]
\author{Tommi A Pirinen \scriptsize \guilsinglleft{}tommi.pirinen@helsinki.fi\guilsinglright{}}
\institute{University of Helsinki\\Department of Modern Languages}
\date{\today}

\begin{document}

\selectlanguage{english}

\HyTitle

\begin{frame}
    \frametitle{Outline}
    \tableofcontents
\end{frame}


\AtBeginSection[]
{
  \begin{frame}<beamer>
    \frametitle{Outline}
    \tableofcontents[currentsection]
  \end{frame}
}

\section{Introduction}

\begin{frame}
    \frametitle{Finite-State Spell-Checking}
    
    \begin{itemize}
        \item A simple task: go through all words in text to see if they
            belong to language LL, if not, modify them with relation
            EE to fit into language LL
        \item In Finite-State world language model LL is any (weighted) single
            tape finite-state automaton recognising the words of the
            language
        \item The error model EE is any (weighted) two-tape automaton, that
            describes spelling errors, i.e. mapping from misspelt word into
            the correct one
        \item In this tutorial we build simple WFST language model and piece
            together error model from different error types stored in FSTs
    \end{itemize}
\end{frame}

\begin{frame}
    \frametitle{(Weighted) finite-state spell-checking graphically}
    From this misspelled word as automaton:\\
\includegraphics{cta}
    Applying this very specific error (correction) model:\\
\includegraphics{cta2cat}
    We can compose or intersect with a word in this dictionary:\\
\includegraphics{catses}
\end{frame}


\begin{frame}
    \frametitle{With HFST Tools (but mostly generic FST algebra)}
    
    \begin{itemize}
        \item The tools we use to build the automata are from HFST project
            \url{http://hfst.sf.net}
        \item but most what I show here is finite-state algebra, which your
            favorite fst tools should support as well
        \item The chain of libraries to get HFST automata spellers to any
            desktop application is HFST 
            ospell$\rightarrow$voikko$\rightarrow$enchant (with few
            exceptional application not supporting this).
            \url{https://kitwiki.csc.fi/twiki/bin/view/KitWiki/HfstUseAsSpellChecker}
    \end{itemize}
\end{frame}

\section{Language Models}

\begin{frame}
    \frametitle{Easy baseline for language model---a corpora of correctly
    spelled text}
    \begin{itemize}
        \item The most basic language model for checking if word is correctly
            written is a word-form list: \texttt{abacus, \ldots, zygotes}, 
            right?
        \item Fast way to build such word-list now: grab an online text
            collection and separate it to a words
        \item Doing it like this, we also get: probabilities!
            $P(w) = \frac{c(w)}{CS}$
        \item In the end we'd probably mangle the probabilities into
            tropical logprobs that our wfst's use by default, say:
            $-\log P(w)$
    \end{itemize}
\end{frame}

\begin{frame}
    \frametitle{Compiling corpus into a spell-checker with HFST}
    \begin{itemize}
        \item A simple single command-line (no line breaks):
            \texttt{tr -d '[:punct:]' < demo-corpus.txt |
            tr -s '[:space:]' '\textbackslash n' |
            sort | uniq -c |
            awk -f tropicalize-uniq-c.awk --assign=CS=20 |
            hfst-txt2fst -f openfst -o demo-lm.hfst}
        \item In order: clean punctuation, tokenize, sort, count, log probs and
            compile
        \item Let's try; you may grab corpus from
            \url{}, or Wikipedia dump,  write one on the fly if you want to
            test it now
    \end{itemize}
\end{frame}

\begin{frame}
    \frametitle{A language model as a FST}
    Resulting language model should look like this:
\includegraphics{demo-lm}
\end{frame}

\begin{frame}
    \frametitle{Other language models}
    \begin{itemize}
        \item A word-form list is only good for very basic demoing for mostly
            isolating languages
        \item Since any given finite-state dictionary is usable, we can 
            use---and have used---e.g.:\begin{itemize}
                \item Xerox style morphologies (lexc, twolc, xfst)
                    compiled with foma or hfst tools
                \item apertium dictionaries
                \item hunspell dictionaries---with a very kludgey collection
                    of conversion scripts
                \item AT\&T / openfst morphologies (are there any?)
            \end{itemize}
        \item from a morphological analyser you may want to project off the
            analysis tape, we currently won't use it
        \item weighting language models afterwards from corpora is simple, cf.
            for example my past publications for HFST scripts of that
    \end{itemize}
\end{frame}

\section{Error Models}

\begin{frame}
   \frametitle{Building error models from scratch}
   \begin{itemize}
       \item Error models are two-tape automata that map misspelled word-forms
           into a correctly spelled one
       \item Basic error types / corrections covered here: typing errors 
           (Levenshtein--Damerau), phonemic/orthographic competence errors,
           common confusables word-forms
       \item Other relatively easy corrections not covered here: phonemic
           folding, errors in context
       \item Since types of errors that can appear are very different, we
           build now separate automata components for each, and then
           connect them together
       \item Here is a good time to look back at the dictionary to see what
           the weights are and scale error weights accordingly using
           the Stetson-Harrison algorithm
   \end{itemize}
\end{frame}

\begin{frame}
    \frametitle{Basic component 1: Run of correctly spelled characters}
    We assume most misspelled words will have most of the letters correct,
    to account these parts we need the identity automaton:\\
    \texttt{echo '?*' | hfst-regexp2fst -o anystar.hfst}\\
    \includegraphics{anystar}
\end{frame}

\begin{frame}
    \frametitle{Basic component 2: Typing error}
    A single typing error, as in popular Damerau-Levenshtein algorithm, is
    a) removal, b) addition or c) change of a character, using fixed alphabet
    for example with a and b, and let's say this error is less unlikely than
    any other for all combinations ( = 10):\\
    \texttt{echo 'a:0::10 | 0:a::10 | a:b::10 | b:0::10 | 0:b::10 | b:a::10' | hfst-regexp2fst -o edit1.hfst}
    \includegraphics{edit1-ab10}
\end{frame}

\begin{frame}
    \frametitle{Excursion 1: Swap in Levenshtein automaton}
    The fourth type of error in edit distance algorithm, transposition or
    swapping adjacent characters, is useful, but makes the automaton a bit messy
    and tedious to write (which is why we have few scripts to generate them).
    This is the additional component, that must be added \emph{per each pair of
    characters in alphabet}:\\
    \texttt{echo 'a:b::10 (b:a) | b:a::10 (a:b) | hfst-regexp2fst}\\
    \includegraphics{edit1swap-ab10}
\end{frame}

\begin{frame}
    \frametitle{Edit Distance Automata}
    \begin{itemize}
        \item Combining the correctly written parts and a single edit gets us
            a nice baseline error model: 
            \texttt{hfst-concatenate anystar.hfst edit1.hfst | 
            hfst-concatenate - anystar.hfst -o errm1.hfst}
        \item Since it's just an automaton, all regular tricks of FST algebra
            just work, particularly:
            \texttt{hfst-repeat -f 1 -t 7 -i errm1.hfst -o errm7.hfst}
        \item (A script to automate writing of the combinations of swaps:
            \texttt{python editdist.py -d 1 -S -a demo.hfst |
            hfst-txt2fst -o edit1.hfst}, from \url{})
    \end{itemize}
    \includegraphics{errm.edit1}
\end{frame}

\begin{frame}
    \frametitle{Common orthography-based competence errors}
    English orthography sometimes leads to errors that are not
    easily solvable by basic edit distance, so we can extend the
    edit distance errors by set of commonly confused substrings (N.B.
    input to strings2fst expects TAB between strings and weight):\\
    \texttt{hfst-strings2fst -j -o en-orth.hfst\\
    of:ough	5
    ier:eir	5}
    \\
    \includegraphics{en-orth}
\end{frame}

\begin{frame}
    \frametitle{Edit Distance Plus Orthographic Errors}
    \begin{itemize}
        \item Combine orthographic errors with edit1 error before making the
            full-word covering error-model:
            \texttt{hfst-disjunct edit1.hfst en-orth.hfst -o edit1+en-orth.hfst}
        \item And rest as before: 
            \texttt{hfst-concatenate anystar.hfst edit1+en-orth.hfst | 
            hfst-concatenate - anystar.hfst -o errm.edit1+en-orth.hfst}
    \end{itemize}
    \includegraphics{errm.edit1+en-orth}
\end{frame}

\begin{frame}
    \frametitle{Confusion set of words}
    Very simple, yet very important piece of error model, common word-specific
    mistakes, of course compiled from a list of mistakes:\\
    \texttt{hfst-strings2fst -j -o en-confusables.hfst\\
    teh:the	2.5
    mispelled:misspelt	2.5
    conscienchous:conscientious	2.5}
    \includegraphics{en-confusables}
\end{frame}

\begin{frame}
    \frametitle{Finally: hacking all these together}
    \begin{itemize}
        \item adding one more component to the mixture; the full-word errors
            can be just disjuncted to any existing full error model:
            \texttt{hfst-disjunct en-confusables errm.edit1+en-orth.hfst
            -o errm.everything.hfst}
        \item should your language happen to use productive compounding you may
            wish to \texttt{hfst-repeat -f 1} the full-word errors before
            disjuncting
    \end{itemize}
    \includegraphics{errm.everything}
\end{frame}

\section{Misc. practicalities}

\begin{frame}
    \frametitle{Metadata}

\end{frame}

\begin{frame}
    \frametitle{Zipping and Installation Location}

\end{frame}

\end{document}
% vim: set spell:

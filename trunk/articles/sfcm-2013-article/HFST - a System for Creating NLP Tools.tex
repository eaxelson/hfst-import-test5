\documentclass{llncs}

\usepackage{llncsdoc}
\usepackage{multirow}
\usepackage{caption}
\usepackage{url}

\usepackage{fontspec}
\usepackage{xunicode}
\usepackage{xltxtra}
\usepackage{expex}

%
\begin{document}
%
\title{HFST---a System for Creating NLP Tools}
%
\author{Krister Lind\'{e}n \and Erik Axelson \and Senka Drobac \and Sam Hardwick \and\\
Tommi A Pirinen \and Miikka Silfverberg \and ...}

\institute{University of Helsinki\\
Department of Modern Languages\\
Unioninkatu 40 A\\
FI-00014 Helsingin yliopisto, Finland\\
\email{\{krister.linden, erik.axelson, senka.drobac, sam.hardwick,\\
tommi.pirinen, miikka silfverberg, ...\}@helsinki.fi}}

\maketitle

% Removed the manual bibstyle in favor of splncs03.bst,
% if it's required fetch it from svn history

\begin{abstract}
%Krister
\keywords{keywords here}
\end{abstract}


\section*{Introduction}
% Krister

\section{Applications and Tests}\label{hfst:structural-layout}

\subsection{Language identification}
\begin{itemize}
\item Language identification is the task of recognizing the language
  of a text excerpt.
\item Lots of application e.g. processing documents written in several
  different languages.
\item Several standard methods including character n-grams \cite{cavnar/1994}.
\item One n-gram model per language. The model giving the highest
  probability wins.
\item Character n-gram models can be implemented as weighted
  finite-state automata.
\item Degree of uncertainty of the guess? 
\item We use hfst-tools for doing this. The underlying mechnism
  utilizes the tagger tools in hfst.
\item Experiment with Finnish, Swedish and English. Add Estonian,
  Spanish, French?
\end{itemize}
% Miikka

\subsection{Morphologies and Guessers}
\label{sec: morph-guessers}
% Juha, Miikka
\begin{itemize}
\item Morphological guessers are needed to compensate for insufficient
  coverage of morphological analyzers.
\item Suffix based guessing has proved sufficient in practice for
  agglutinating languages like Finnish~\cite{linden/2009/nodalida}.
\item Using hfst tools it is possible to transform an existing morphological analyzer into a guesser. 
\item If the analyzer marks declensions of words, the guesser will
  guess the declension in addition to the inflectional information.
\item In addition the guesser constructs a stem for the word.
\item If the morphological analyzer is weighted, the weights are used
  to arrange guesses in order of likelihood. The guesser also prefers
  longer matches with words recognized by the analyzer to shorter
  ones.
\item Using information about declension classes also helps in
  guessing. Some declensions might only contain a couple of words,
  implying that an unknown word is unlikely to belong to that
  declension. In addition it might be known that new words are likely
  to be added to a small set of declension classes.
\item The guesser is a convenient tool when extending the lexicon.
\item Some results for Finnish.
\end{itemize}

\subsection{Spell-checking}
% Tommi

Weighted finite-state methods in performing spell-checking and correction is a
relatively recent branch of study in research of spell-checking. The concept is
simple: finite-state morphological analysers and such can be trivially ported
into spell-checking dictionaries providing a language model for the correctly
spelled words in the spell-checking system. A baseline finite-state model for
correcting spelling errors can be inferred from the language model by creating
a Levenshtein-Damerau automaton based on the alphabetic characters present in
the language. The language model can be simply trained to prefer more common
words when the Levenshtein-Damerau distance between to suggestions is the same.
This is done by basic unigram language model training that simply maximises
the frequency of the suggested word. To our experience even relatively moderate
training material will gain improvement in quality as the statistical training
improves the discriminative power of the model, and the likelihood of random
typing error is more likely in frequent words.

The practical process of creating a finite-state spell-checker and corrector
is really simple: given an analysator capable of recognising correctly spelled
word-forms of a language, take a projection to the surface forms to create a
single-tape automaton. The automaton is trained with corpus word-form list, 
where end-weight of each word-form is e.g. $-\log\frac{c(wf)}{CS}$, where 
$c(wf)$ is the count of word-forms, and $CS$ is the corpus size. Words not
found in the corpus are given some weight $w_{max} > -\log\frac{1}{CS}$ to
push them in the bottom of the suggestion list; this weighting can be done
in finite-state algebra by composition of weighted $\Sigma^{\star}$ language,
or by manually fixing the data structure.

The error model can be improved from the baseline Levenshtein-Damerau distance
metric as well. For this purpose we need an error corpus, that is, set of
errors with their frequencies. This can be semi-automatically extracted from
weakly annotated sources, such as Wikipedia. From wikipedia we get, among tons
of other things, word-to-word corrections. It is possible to use the specific
word-to-word corrections to create simple extension of common confusables to
error model. Another way is to re-align the corrections using the
Damerau-Levenshtein algorithm, and train the original distance measure with
frequencies of the corrections in same manner as we did for word-forms above.

As an example of simplicity of this process, we have obtained an open source
German morphological analyser
morphisto~\footnote{\url{http://code.google.com/p/morphisto/}} to generate a
spelling checker, trained it with word-forms extracted from German
Wikipedia~\footnote{\url{http://de.wikipedia.org}} and applied it to Wikipedia
data to find spelling errors and correct them with n~\% of precision. The whole
script to do this is in our version control~\footnote{\url{}}, and it took us
no more than one work day by one researcher to implement this application.

\subsection{Named Entity Recognition}
% Jyrki, Juha (Sam?)

\subsection{Language Generation for Unknown words}
\begin{itemize}
\item In language generation the goal is to generate automatic
  messages about a computer system in natural language.
\item The system includes some concepts like weather phenomena or
  airline connections and it generates messages about the state of the
  concepts for a user.
\item For agglutinating languages this presents n challenge because
  the base forms of the concepts to be inflected depending on syntactic
  context.
\item A generation system can utilize a morphological analyzer, but
  all of the concepts might not be recognized by the analyzer.
\item Therefore another mechanism is needed for generating word forms
  from the lemma of the base form of the concept.
\item Morphological guessers presented in~\ref{sec: morph-guessers}
  can be used for generation of word forms in addition to analysis.
\item When the word class and base form of the word are known, a
  paradigm of word forms can be generated.
\item This can also be used when expanding a morphological
  analyzer. It can be used to generate a list of suggestion forms,
  which can be examined quickly to determine whether a word could
  belong to a given declension class.
\item Example from Finnish.
\end{itemize}

\section{Examples for User Environments}

\subsection{An Interface in Python}
% Erik, Tommi

In addition to API library and command line tools, HFST library can also be used
through SWIG-generated python bindings. Currently the bindings are offered for
python programming language versions 2 and 3. All HFST functionalities are available
via both versions, but the python interpreters themselves have some differences.
Python 2 allows HFST exceptions to be caught directly, but python 3 requires the use
of a special wrapper function written as a part of the bindings. On the other hand, 
python 3 has better support for unicode characters, so it is probably a better choice
for linguistic applications.

The python bindings in particular make it smooth to use language models
developed for HFST in rapid prototyping of advanced tools. For an example, in
my current project I developed a chunker for Finnish language by simply
bracketing adjacent agreeing cases and few other similar expressions with
basically few lines of code on top of existing morphological analysers. E.g.
given Finnish sentence ``miljoona kärpästä voi olla väärässä paikassa'' we get
bracketing all three phrases bracketed by really simple python based bracketing
that are illustrated in following figure gloss:

\ex
\begingl
\gla Miljoona$_1$ kärpästä$_1$ voi$_2$ olla$_2$ väärässä$_3$ paikassa$_3$//
\glb Million-{\sc Num} fly-{\sc Par} can-{\sc AuxV} be-{\sc Inf} wrong-{\sc Ine} place-{\sc Ine}//
\glft `Million flies can be in a wrong place' //
\endgl
\xe


\subsubsection{A Chatroom Morphology tool}
% Sam

\subsection{HFST on Unix, Mac and Windows}
% Erik, Tommi

Portability has been one of the design goals of HFST system. The current 
versions are available or compilable on all POSIX-supporting systems, including
Linuxes, Mac OS X and Cygwin under Windows system. Compilation is also possible 
on MinGW under Windows. Another layer of portability
is attained by HFST's programming language bindings via SWIG, it is possible to
use full HFST library through it's SWIG interfaces in all supported systems with
python versions 2 and 3.


\subsection{Other user-oriented items}
% Erik

\section{Under the Hood}

\subsection{An independent XFST module}
% Erik (Erik)

HFST command line tools include an XFST parser tool that can be used in interactive
mode or to compile scriptfiles. The tool implements the same functionalities as the
original XFST (Xerox Finite-State Tool) which is a general-purpose utility for computing
with finite-state networks.


\subsection{XFST regexp compilation}
% Senka, Miikka

\subsection{Pmatch with applications for NER}
% Sam

\subsection{Other HFST development tools}
% Erik

\section{Discussion}\label{hfst:discussion}

\subsection{HMM vs. CRF for English and Finnish taggers}
% Miikka

\subsection{Adding rules to statistical systems (taggers and parser)}
% Miikka

\section{Conclusion}\label{hfst:conclusion}

\subsubsection*{Acknowledgments}

\bibliographystyle{splncs03}
\bibliography{sfcm-2013}

\end{document}
% vim: set spell:

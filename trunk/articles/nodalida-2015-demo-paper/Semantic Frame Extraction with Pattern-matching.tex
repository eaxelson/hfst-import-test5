\documentclass[11pt]{article}
\usepackage{nodalida2015}
\usepackage{times}
\usepackage{mathptmx}
%\usepackage{txfonts}
\usepackage{url}
\usepackage{latexsym}
\usepackage[T1]{fontenc}
\usepackage[utf8]{inputenc}
\usepackage{amsmath}
\usepackage{framed}
\special{papersize=210mm,297mm} % to avoid having to use "-t a4" with dvips 
%\setlength\titlebox{6.5cm}  % You can expand the title box if you really have to

\title{Extracting Semantic Frames using hfst-pmatch}

\author{Sam Hardwick \\
  University of Helsinki \\
  {\tt sam.hardwick} \\ {\tt @iki.fi} \\\And
  Miikka Silfverberg \\
  University of Helsinki \\
  {\tt miikka.silfverberg} \\ {\tt @iki.fi}\\\And
  Krister Lind{é}n \\
  University of Helsinki \\
  {\tt krister.linden} \\ {\tt @helsinki.fi}}

\date{}

\begin{document}
\maketitle
\begin{abstract}
  We use \verb+hfst-pmatch+~(\newcite{pmatch}), a
  pattern-matching tool mimicking and extending Xerox
  \verb+fst+~(\newcite{karttunen/2011}), for demonstrating how to develop a
  semantic frame extractor. We
  select a FrameNet~(\newcite{framenet}) frame and write shallowly syntactic pattern-matching rules
  based on part-of-speech information and morphology from either a morphological automaton or
  tagged text.
\end{abstract}

\section{Introduction}
\verb+pmatch+ is a pattern-matching operation for text based on regular
expressions. It uses a context-free grammar on regular expression terminals,
which allows for recursive, self-referencing rules which would not be possible
in a fully regular formalism. The matched patterns may be efficiently tagged,
extracted and modified by the rules.

Large-scale named-entity recognisers (NERs) have been developed in
\verb+pmatch+ for Swedish and Finnish. Here we demonstrate a bottom-up approach
to using it to identify the frame ``Size'' in FrameNet.

\section{A Semantic Frame}

A semantic frame~(\newcite{semantic-frame}) is a description of a \emph{type} of event, relation or entity
and related participants. For example, in Framenet,
a database of semantic frames,
the description of an \verb+Entity+ in terms of physical space occupied by it is
an instance of the semantic frame \verb+Size+. The frame is evoked by
a lexical unit (LU), also known as a frame evoking element (FEE), which is a
word (in this case an adjective)
such as ``big'' or ``tiny'', descriptive of the size of the \verb+Entity+.
Apart from \verb+Entity+, which is a core or compulsory element, the
frame may identify a \verb+Degree+ to which the \verb+Entity+ deviates
from the norm (``a \textbf{really} big dog'') and a \verb+Standard+ to
which it is compared (``tall \textbf{for a jockey}'').


  \begin{center}
\begin{table}[h]
  \begin{tabular}{ | l | p{4cm} |}
\hline
Lexical Unit (LU) & Adjective describing magnitude (large, tiny, ...) \\
\hline
Entity (E) & That which is being described (house, debt, ...) \\
\hline
Degree (D), optional & Intensity or extent of description
(really, quite, ...) \\
\hline
Standard (S), optional & A point of comparison (for a jockey, ...) \\
\hline
    \end{tabular}
    \caption{The semantic frame \emph{Size}.}
\end{table}
      \end{center}

For example:

\begin{table}[h]
$$
\Big[_\text{Size}\Big[_\text{E}\text{He} \Big]
  \text{is} \Big[_\text{D} \text{quite} \Big] \Big[_\text{LU}\text{tall} \Big]
  \Big[_\text{S} \text{for a jockey} \Big] \Big]
$$
\caption{A tagged example of \emph{Size}}
\end{table}

\section{A Rule}

A \verb+pmatch+ ruleset consists of a number of named regular expressions and
functions, exactly one of which is the top-level rule which is named \verb+TOP+
or is introduced with the directive \verb+regex+. For example:

\begin{table}[h]
\small
\begin{framed}
\begin{verbatim}
define my_colours {green} | {red};
define TOP my_colours EndTag(colour);
\end{verbatim}
\end{framed}
\normalsize
\caption{Introducing \texttt{pmatch} syntax}
\end{table}

The effect of the directive \verb+EndTag()+ is to tag whatever is matched by its
rule (here shown with an unintentional effect):

%\begin{table}[h]
  \small
  \begin{framed}
\begin{verbatim}
The light went <colour>green</colour>
and the mechanism was
trigge<colour>red</colour>.
\end{verbatim}
\end{framed}
  \normalsize
%  \caption{Demonstrating \texttt{pmatch} output}
  %  \end{table}

To avoid tagging the ``red'' at the end of ``triggered'', we need to add a word
boundary to the rule. This could be accomplished by defining eg.\@ \verb+W+ to
be whitespace (\verb+Whitespace+), punctuation (\verb+Punct+) or the limit of
input (\verb+#+). \verb+W+ may then be interpolated
in rules when we want to capture whitespace inside the pattern, or checked for
with run-time context checking just to make sure there is a word boundary at the
edge of our rule (\verb+LC()+ and \verb+RC()+ check left and right contexts
respectively).

A simple and common syntactic realisation of the \verb+Size+ frame is a single
noun phrase containing one of the LUs, such as
``the big brown dog that ran away''. Here we'd like to identify ``big'' as \verb+LU+,
``brown dog'' as \verb+Entity+ and the combination as \verb+Size+.
Our first rule for identifying this type of construction might be

\begin{table}[h]
  \small
  \begin{framed}
\begin{verbatim}
define LU {small} | {large} |
  {big} EndTag(LU);
define Size1 LU (Adjective)
  Define(Noun EndTag(Entity));
define TOP Size1 EndTag(Size);  
\end{verbatim}
\end{framed}
\normalsize
\caption{A simplified first rule}
\end{table}

This ruleset has been simplified for brevity -- it has only a few of the
permitted LUs, and word boundary issues have not been addressed.

The \verb+Define()+ inside the definition of \verb+TOP+ is an anonymous rule
that delimits the effect of the \verb+EndTag+. The extra \verb+Adjective+ is
optional, which is conveyed by the surrounding parentheses.

We can verify that our rules extract instances of our desired pattern by compiling
them with \verb+hfst-pmatch2fst+ and running the compiled result with
\verb+hfst-pmatch --extract-tags+. In the following we have
inputted the text of the King James Bible from Project
Gutenberg (\url{gutenberg.org}) and added some extra characters on both
sides for
a concordance-like effect:

%\begin{table}[h]
\hfill \break
  \small
  \begin{framed}
\begin{verbatim}
...
there lay a <Size><LU>small</LU>
round <Entity>thing</Entity></Size>
...
there was a <Size><LU>great</LU>
<Entity>cry</Entity></Size> in Egypt
...
saw that <Size><LU>great</LU>
<Entity>work</Entity></Size> which
...
\end{verbatim}
\end{framed}
\normalsize
%\caption{Fragments of tagged running text}
%  \label{bibletext}
%\end{table}

\verb+pmatch+ may be operated in various modes. In \verb+locate+ mode the
position and length of each match is given, and only the outermost tag is
supplied. \verb+match+ mode (which is the default) tags and outputs running
text, and \verb+extract+ mode does the same but omits parts of the input that
aren't matched by a rule. Matches may also be extracted via an API call, for
example in order to achieve the above-seen concordance effect.

A natural next step is to add optional non-core elements, such as an adverb
preceding the LU being tagged as \verb+Degree+ and a noun phrase beginning with
``for a'' following it as \verb+Standard+.

\begin{table}[h]
  \small
  \begin{framed}
\begin{verbatim}
define Size1
  Define(Adverb EndTag(Degree))
  LU (Adjective)
  Define(Noun EndTag(Entity))
  Define({for a} NP EndTag(Standard));
\end{verbatim}
\end{framed}
\normalsize
\caption{Extending the rule with optional elements}
\end{table}

Here are some examples this rule finds in the British National
Corpus~(\newcite{bnc}).

%\begin{table}[h]
  \small
  \begin{framed}
\begin{verbatim}
...
presence of an <Size>
  <Degree>arbitrarily</Degree>
  <LU>small</LU> <Entity>
  amount</Entity></Size> of dust
...
one <Size><LU>small</LU>
  <Entity>step</Entity>
  <Standard>for a man</Standard>
  </Size>
...
\end{verbatim}
  \end{framed}
  \normalsize
%  \caption{Tagged text with optional elements}
%  \label{bnctext}
%  \end{table}

We can see that in ``small amount of dust'' we might want to
tag not just the immediate noun as \verb+Entity+ but the entire noun phrase
(which could be implemented up to a context-free definition of a noun phrase),
and in ``one small step for a man'' a common indirect use of the \verb+Standard+
construction.

The FrameNet corpus itself is a
good source for finding more cases.

As well as correct matches, such as ``small round thing'' in the biblical
example, we have metaphorical meanings of \verb+Size+, such as ``great cry''.
This may or may not be desired -- perhaps we wish to do further processing to
identify the target domains of such metaphors, or perhaps we wish to be able
to annotate physical size and physical size only.

\subsection{Incorporating Semantic Information}

Size is a very metaphorical concept, and syntactic rules as above will produce a large amount of matches that relate to such uses, eg. ``a great cry'' or ``a big deal''. If we wish to refine our rules to detect such uses, there are a few avenues for refinement.

First of all, some LUs are much more metaphorical than others. A ``great man'' is almost certainly a metaphorical use, whereas a ``large man'' is almost certainly concrete. Accuracy may be improved by requiring ``great'' to be used together with a common concrete complement, like ``great crowd''. This is a matter of running the rules on a corpus and doing some statistics on the result.

There are also semantic classifications of words, such as WordNet(~\newcite{wordnet}). We may compile the set of hyponyms of \emph{physical entity} and require them to appear as the nouns in our rules.

\begin{table}[h]
\small
\begin{framed}
\begin{verbatim}
define phys_entity
  @txt"phys_entity.txt";
! a list of singular baseforms
! can be expanded to include
! eg. plurals by suitably composing
! it with a dictionary automaton
\end{verbatim}
\end{framed}
\normalsize
\caption{Reading an external linguistic resource}
\end{table}

\subsection{Incorporating Part-of-speech Information}

We have hitherto used named rules for matching word classes, like \verb+Noun+,
without specifying how they are written. Even our collection of LUs might need
some closer attention -- for example ``little'' could be an adverb.

Considering that in writing our
rules we are effectively doing shallow syntactic parsing, even a very simple
way to identify parts of speech may suffice: a morphological dictionary.
For example, a finite-state transducer representing English morphology may be
used to define the class of common nouns as in table \ref{dictrules}.

\begin{table}[h]
\small
  \begin{framed}
\begin{verbatim}
! The file we want to read
define English @bin"english.hfst";
! We compose it with a noun filter
! and extract the input side
define Noun English .o.
 [?+ "<NN1>" | "<NN2>"].u;
! (NN1 is singular, NN2 plural)
\end{verbatim}
\end{framed}
  \normalsize
  \caption{Using a dictionary to write POS rules}
  \label{dictrules}
  \end{table}

If we have the use of a part-of-speech tagger, we may write our rules to act
on its output, as in table \ref{posrules}.

\begin{table}[h]
  \small
  \begin{framed}
\begin{verbatim}
define Noun LC(W) Wordchar+
 ["<NN1>"|"<NN2>"] RC(W);
\end{verbatim}
\end{framed}
  \normalsize
  \caption{Using tags in pre-tagged text}
  \label{posrules}
  \end{table}

\section{Increasing Coverage}
Having considered for each rule where \verb+Degree+ and \verb+Standard+ may occur, coverage may be evaluated by also finding those cases where a LU is used as an adjective but hasn't been tagged, eg.

\small
\begin{verbatim}
define TOP Size1 | Size2 | ...
 Define(LU EndTag(NonmatchingLU));
\end{verbatim}
\normalsize

The valid match is always the longest possible one, so \verb+NonmatchingLU+ will be the tag only if no subsuming \verb+SizeN+ rule applies.

For example in

\small
\begin{framed}
\begin{verbatim}
the moving human body is
<NonmatchingLU>large</NonmatchingLU>
obtrusive and highly visible
\end{verbatim}
\end{framed}
\normalsize

We see another realisation of the frame: the \verb+Entity+ being followed by a
copula, and the \verb+LU+ appearing to the right. We could write the rule
\verb+Size2+ to capture this, adding positions for non-core elements either by
linguistic reasoning or by searching the corpus.

% If you use BibTeX with a bib file named eacl2014.bib, 
% you should add the following two lines:
\bibliographystyle{acl}
\bibliography{nodalida2015}

\end{document}

% This is the Authors' notes demonstration file with content stripped out
% and sections from Krister's email substituted.
% 
% It might be a good idea to verify that Authors' Instructions.pdf
% (in this directory) is adhered to before submitting.
%
% llncs.doc is actually a tex file, it's the source for the Authors'
% Instructions. There are good examples of eg. tables and graphs there.
\documentclass{llncs}
\usepackage{llncsdoc}
%
\begin{document}
%
\title{HFST - Framework for Compiling and Applying Morphologies}
%
\author{Author Names\inst{}}

\institute{University of Helsinki}

\maketitle
%
% Modify the bibliography environment to call for the author-year
% system. This is done normally with the citeauthoryear option
% for a particular contribution.

% SH: Not sure we need this ... the example file had two articles, the latter with this format.
% It should be possible to comment out until the abstract and get the 'regular' citation.
\makeatletter
\renewenvironment{thebibliography}[1]
     {\section*{\refname}
      \small
      \list{}%
           {\settowidth\labelwidth{}%
            \leftmargin\parindent
            \itemindent=-\parindent
            \labelsep=\z@
            \if@openbib
              \advance\leftmargin\bibindent
              \itemindent -\bibindent
              \listparindent \itemindent
              \parsep \z@
            \fi
            \usecounter{enumiv}%
            \let\p@enumiv\@empty
            \renewcommand\theenumiv{}}%
      \if@openbib
        \renewcommand\newblock{\par}%
      \else
        \renewcommand\newblock{\hskip .11em \@plus.33em \@minus.07em}%
      \fi
      \sloppy\clubpenalty4000\widowpenalty4000%
      \sfcode`\.=\@m}
     {\def\@noitemerr
       {\@latex@warning{Empty `thebibliography' environment}}%
      \endlist}
      \def\@cite#1{#1}%
      \def\@lbibitem[#1]#2{\item[]\if@filesw
        {\def\protect##1{\string ##1\space}\immediate
      \write\@auxout{\string\bibcite{#2}{#1}}}\fi\ignorespaces}
\makeatother
%
\begin{abstract}
% KL
\keywords{finite-state morphology}
\end{abstract}

\section*{Introduction}
% KL


\section{Structural Layout}
% MS


\section{Structural Layout cont.}
% SH
% Comment to remove before submission:
% the relatively insignificant material here should really be incorporated into
% another section..
There has been considerable progress in achieving HSFT's goal of acting as a
compatibility layer between different representations of finite-state
transducers and, more importantly, the operations and formalisms (eg.
\verb+lexc/twolc+ , \verb+xfst+ , \verb+sfst+) that have been implemented
for them. HFST is now independent of any particular library and requires no
custom extensions to the libraries it uses.

\subsection{Dynamic Linking to Underlying Libraries}
Previously HFST relied on custom extensions to the libraries it supported,
namely OpenFst and SFST, and it was necessary to statically link them into
\verb+libhfst+. This was obviously also rather restrictive in terms of
new versions and different use cases (eg. experimentation with local changes to
the underlying libraries.

HFST3 supports conditional compilation of all its elements that provide
interfaces to underlying libraries, and dynamic linking is done to whichever
libraries the user configures HFST3 to use.

\subsection{Stand-alone Use and Flexibility}
Due to these improvements, HFST3 can also be built without any external
libraries (in which case only the operations for a simple internal
representation and optimized-lookup (\ref{optimized-lookup}), building
from text representation and fast lookup will be supported) or only the user's
self-made library, which will then support compilation to optimized-lookup
format.

\section{Coding Priciples}
% EA


\section{Data Format}
% EA

\section{Alphabet}
% EA

\section{Algorithmic Improvements}
% MS

\section{Xerox Compatibility}
% TP

\section{Compilation Performance}
% EA

\section{Application Areas}
% TP

\section{Optimized Lookup}\label{optimized-lookup}
% SH

\section{Discussion}
%KL

\section{Conclusion}
% KL

%
%
%
% ---- Bibliography ----
%
\begin{thebibliography}{}
% An example from the demo:
%\bibitem[1980]{2clar:eke}
%Clarke, F., Ekeland, I.:
%Nonlinear oscillations and
%boundary-value problems for Hamiltonian systems.
%Arch. Rat. Mech. Anal. 78, 315--333 (1982)

\end{thebibliography}
\end{document}

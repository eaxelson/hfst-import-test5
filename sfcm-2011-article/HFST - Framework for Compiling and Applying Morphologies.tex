% This is the Authors' notes demonstration file with content stripped out
% and sections from Krister's email substituted.
% 
% It might be a good idea to verify that Authors' Instructions.pdf
% (in this directory) is adhered to before submitting.
%
% llncs.doc is actually a tex file, it's the source for the Authors'
% Instructions. There are good examples of eg. tables and graphs there.
\documentclass{llncs}
\usepackage{llncsdoc}
\usepackage{multirow}
\usepackage{caption}
%
\begin{document}
%
\title{HFST - Framework for Compiling and Applying Morphologies}
%
\author{Author Names\inst{}}

\institute{University of Helsinki}

\maketitle
%
% Modify the bibliography environment to call for the author-year
% system. This is done normally with the citeauthoryear option
% for a particular contribution.

% SH: Not sure we need this ... the example file had two articles, the latter with this format.
% It should be possible to comment out until the abstract and get the 'regular' citation.
\makeatletter
\renewenvironment{thebibliography}[1]
     {\section*{\refname}
      \small
      \list{}%
           {\settowidth\labelwidth{}%
            \leftmargin\parindent
            \itemindent=-\parindent
            \labelsep=\z@
            \if@openbib
              \advance\leftmargin\bibindent
              \itemindent -\bibindent
              \listparindent \itemindent
              \parsep \z@
            \fi
            \usecounter{enumiv}%
            \let\p@enumiv\@empty
            \renewcommand\theenumiv{}}%
      \if@openbib
        \renewcommand\newblock{\par}%
      \else
        \renewcommand\newblock{\hskip .11em \@plus.33em \@minus.07em}%
      \fi
      \sloppy\clubpenalty4000\widowpenalty4000%
      \sfcode`\.=\@m}
     {\def\@noitemerr
       {\@latex@warning{Empty `thebibliography' environment}}%
      \endlist}
      \def\@cite#1{#1}%
      \def\@lbibitem[#1]#2{\item[]\if@filesw
        {\def\protect##1{\string ##1\space}\immediate
      \write\@auxout{\string\bibcite{#2}{#1}}}\fi\ignorespaces}
\makeatother
%
\begin{abstract}
% KL
\keywords{finite-state morphology}
\end{abstract}

\section*{Introduction}
% KL


\section{Structural Layout}
% MS


\section{Structural Layout cont.}
% SH
% Comment to remove before submission:
% the relatively insignificant material here should really be incorporated into
% another section..
There has been considerable progress in achieving HSFT's goal of acting as a
compatibility layer between different representations of finite-state
transducers and, more importantly, the operations and formalisms (eg.
\verb+lexc/twolc+, \verb+xfst+, \verb+sfst+) that have been implemented
for them. HFST is now independent of any particular library and requires no
custom extensions to the libraries it uses.

\subsection{Dynamic Linking to Underlying Libraries}
Previously HFST relied on custom extensions to the libraries it supported,
namely OpenFst and SFST, and it was necessary to statically link them into
\verb+libhfst+. This was obviously also rather restrictive in terms of
new versions and different use cases (eg. experimentation with local changes to
the underlying libraries.

HFST3 supports conditional compilation of all its elements that provide
interfaces to underlying libraries, and dynamic linking is done to whichever
libraries the user configures HFST3 to use.

\subsection{Stand-alone Use and Flexibility}
Due to these improvements, HFST3 can also be built without any external
libraries (in which case only the operations for a simple internal
representation and optimized-lookup (see section \ref{optimized-lookup}),
building from text representation and fast lookup will be supported) or only
the user's self-made library, which will then support compilation to
optimized-lookup format.

\section{Coding Priciples}
% EA


\section{Data Format}
% EA

\section{Alphabet}
% EA

\section{Algorithmic Improvements}
% MS

\section{Xerox Compatibility}
% TP

\section{Compilation Performance}
% EA

The performance of HFST has improved from version 2.0 to 3.0. 
We compiled two finite-state morphologies in SFST programming language format
with HFST versions 2.0 and 3.0. 
The morphologies were OMorFi (ref.) for Finnish and Morphisto (ref.) for German.
In table~\ref{tab:compilation_times} are the compilation times 
for both morphologies with 
different backend implementations with both versions of HFST. 
Note that the foma implementation was not available in version 2.0.

\begin{table}
\centering
  \begin{tabular}{ c | c | c | c }
  \multicolumn{4}{c}{Compilation times} \\ \hline
  Backend & version & Finnish & German \\ \hline
  \multirow{2}{*}{SFST} & 2.0 & 25:16 & 107:47 \\
  & 3.0 & 5:02 & 6:39 \\ \hline
  \multirow{2}{*}{OpenFst} & 2.0 & 7:54 & 6:23 \\
  & 3.0 & 6:51 & 6:28 \\ \hline
  \multirow{2}{*}{foma} & 2.0 & - & - \\
  & 3.0 & 1:49 & 1:29 \\
  \end{tabular}
  \caption{Compilation times for Finnish and German morphologies with
    HFST. The times are expressed in minutes and seconds.}
  \label{tab:compilation_times}
\end{table}


We can clearly see that the compilation time has improved dramatically
for the SFST implementation.
This is mainly because the new version of SFST, 1.4.2, uses Hopcroft's 
minimization algorithm instead of Brzozowski's. 
We noticed how the minimization algorithm affects performance
already when we were testing HFST version 2.0; 
OpenFst was clearly faster because it uses Hopcroft's algorithm. 
Based on this observation, Helmut Schmid could improve his SFST by 
writing a minimization function that uses Hopcroft's algorithm.

When comparing the compilation times for OpenFst, we see that the
Finnish morphology compiles faster but the German one slightly slower
on HFST version 3.0 than on version 2.0. This is because there are two
factors that contribute to the difference in performance. Firstly, we
are currently using OpenFst version 1.2.7 that is faster than the
previous versions. Secondly, in HFST version 3.0 we no longer use the
same number-to-symbol encodings for all transducers during the same
session. Every time we perform a binary operation on two transducers,
we must harmonize the encodings of the transducers. Nevertheless, 
it seems that the newer, more efficient version of
OpenFst compensates well for this slowness caused by harmonization. 

We did not have the foma implementation available in HFST version 2.0,
but it is evident that it is much faster than the other
implementations in either version of HFST. Foma does not either use
the same symbol-to-number encodings in its transducers, but it still
performs well. It is presumable that symbol harmonization is not a big
factor in the compilation times of morphologies. 


\section{Application Areas}
% TP

\section{Optimized Lookup}\label{optimized-lookup}
% SH

\section{Discussion}
%KL

\section{Conclusion}
% KL

%
%
%
% ---- Bibliography ----
%
\begin{thebibliography}{}
% An example from the demo:
%\bibitem[1980]{2clar:eke}
%Clarke, F., Ekeland, I.:
%Nonlinear oscillations and
%boundary-value problems for Hamiltonian systems.
%Arch. Rat. Mech. Anal. 78, 315--333 (1982)

\end{thebibliography}
\end{document}

% This is the Authors' notes demonstration file with content stripped out
% and sections from Krister's email substituted.
\documentclass{llncs}
%
\usepackage{makeidx}  % allows for indexgeneration
%
\begin{document}
%
\title{HFST - Framework for Compiling and Applying Morphologies}
%
\author{Author Names\inst{}}

\institute{University of Helsinki}

\maketitle
%
% Modify the bibliography environment to call for the author-year
% system. This is done normally with the citeauthoryear option
% for a particular contribution.

% SH: Not sure we need this ... the example file had two articles, the latter with this format.
% It should be possible to comment out until the abstract and get the 'regular' citation.
\makeatletter
\renewenvironment{thebibliography}[1]
     {\section*{\refname}
      \small
      \list{}%
           {\settowidth\labelwidth{}%
            \leftmargin\parindent
            \itemindent=-\parindent
            \labelsep=\z@
            \if@openbib
              \advance\leftmargin\bibindent
              \itemindent -\bibindent
              \listparindent \itemindent
              \parsep \z@
            \fi
            \usecounter{enumiv}%
            \let\p@enumiv\@empty
            \renewcommand\theenumiv{}}%
      \if@openbib
        \renewcommand\newblock{\par}%
      \else
        \renewcommand\newblock{\hskip .11em \@plus.33em \@minus.07em}%
      \fi
      \sloppy\clubpenalty4000\widowpenalty4000%
      \sfcode`\.=\@m}
     {\def\@noitemerr
       {\@latex@warning{Empty `thebibliography' environment}}%
      \endlist}
      \def\@cite#1{#1}%
      \def\@lbibitem[#1]#2{\item[]\if@filesw
        {\def\protect##1{\string ##1\space}\immediate
      \write\@auxout{\string\bibcite{#2}{#1}}}\fi\ignorespaces}
\makeatother
%
\begin{abstract}
% KL
\keywords{finite-state morphology}
\end{abstract}

\chapter*{Introduction}
% KL


\chapter*{Structural Layout}
% MS


\section*{Structural Layout cont.}
% SH
The goal of the HSFT distribution is to act as a transparent compatibility
layer between different representations of finite-state transducers and,
more importantly, the operations and formalisms (eg.
\begin{verbatim}lexc/twolc\end{verbatim}, \begin{verbatim}xfst\end{verbatim},
\begin{verbatim}sfst\end{verbatim} end that have been implemented
for them.

\section*{Coding Priciples}
% EA


\section*{Data Format}
% EA

\section*{Alphabet}
% EA

\section*{Algorithmic Improvements}
% MS

\section*{Xerox Compatibility}
% TP

\section*{Compilation Performance}
% EA

\section*{Application Areas}
% TP

\section*{Optimized Lookup}
% SH

\section*{Discussion}
%KL

\section*{Conclusion}
% KL

%
%
%
% ---- Bibliography ----
%
\begin{thebibliography}{}
% An example from the demo:
%\bibitem[1980]{2clar:eke}
%Clarke, F., Ekeland, I.:
%Nonlinear oscillations and
%boundary-value problems for Hamiltonian systems.
%Arch. Rat. Mech. Anal. 78, 315--333 (1982)

\end{thebibliography}
\clearpage
\addtocmark[2]{Author Index} % additional numbered TOC entry
\renewcommand{\indexname}{Author Index}
\printindex
\clearpage
\addtocmark[2]{Subject Index} % additional numbered TOC entry
\markboth{Subject Index}{Subject Index}
\renewcommand{\indexname}{Subject Index}
\input{subjidx.ind}
\end{document}

\documentclass{llncs}
\usepackage{llncsdoc}
\usepackage[utf8x]{inputenc}
\usepackage{multirow}
\usepackage{caption}
\usepackage{url}
%
\begin{document}
%
\title{HFST---Framework for Compiling and Applying Morphologies}
%
\author{Krister Lind\'{e}n \and Erik Axelson \and Sam Hardwick \and\\
Tommi A Pirinen \and Miikka Silfverberg}

\institute{University of Helsinki\\
Department of Modern Languages\\
Unioninkatu 40 A\\
FI-00014 Helsingin yliopisto\\
\email{\{krister.linden, erik.axelson, sam.hardwick,\\
tommi.pirinen, miikka silfverberg\}@helsinki.fi}}

\maketitle

% Removed the manual bibstyle in favor of splncs03.bst,
% if it's required fetch it from svn history

\begin{abstract}
HFST--Helsinki Finite-State Technology\footnote{hfst.sf.net}
is a framework for compiling and applying linguistic descriptions with
finite-state methods. 
HFST currently collects some of the most important finite-state tools for creating
morphologies and spellers into one open-source platform and supports
extending and improving the descriptions with weights to accommodate the modeling of
statistical information.
HFST offers a path from language descriptions to efficient language applications 
in key environments and operating systems. 
HFST also provides an opportunity to exchange transducers between 
different software providers in order to get the best out of each library. 
% When the descriptions have been compiled into a transducer, 
% they can be exchanged and enhanced with code, corpora and descriptions from other 
% developers and providers using the HFST toolkit.
% Extensive finite-state morphologies are
% available for many languages and spellers for close to a hundred languages from
% all over the world.
\keywords{finite-state libraries, finite-state morphology, natural language applications}
\end{abstract}


\section*{Introduction}
Including language technology in ordinary software applications can be
both laborious and expensive if every language needs its own piece of software
in addition to its lexicons and grammars. Standard interfaces
to language modules have long been an effort equally promoted and hampered 
by large organizations by each having their own standard.
However, with finite-state technology it is possible to go further and encode 
a range of language modules as finite-state transducers with a unified
access interface.

To create the transducers, we still need lexicons and grammars for each language.
Lexicons and grammars exist for close to a hundred languages with various degrees 
of coverage and elaboration. There are ongoing efforts to collect and list available 
sources in various software and data registries, e.g. VLO\footnote{www.clarin.eu/vlo/}, 
META-SHARE\footnote{www.meta-net.eu/meta-share} as well as more specific
efforts for listing open-source finite-state descriptions on the 
HFST--Helsinki Finite-State Technology web site\footnote{hfst.sf.net}.

% 1. Structural Layout \ref{hfst:structural-layout}

Finite-state transducer (FST) technology has been well-known for several 
decades, and many packages of FST calculus exist. Some of them are 
available as open source. If they lose support, changing to another may
easily mean redeveloping the language description. 

The primary goals of HFST are to unify the field and create a framework for developing, 
compiling and applying morphologies, to create convergence and cooperation within the
community which develops finite-state calculus and tools, to create a neutral platform 
where different implementations of the finite-state calculus can coexist and compete 
with each other, and to create a critical mass of research for improving the basic algorithms 
of the calculus, and compilation algorithms.  
HFST does this by providing an interface to an increasing number of software libraries 
for processing finite-state transducers in specialized ways. As far as possible, 
we have tried to avoid implementing yet another finite-state calculus. We have 
rather utilized existing free open source implementations, e.g. SFST by Helmut 
Schmid~\cite{schmid/2005} and foma by Måns Huldén~\cite{hulden/2009} 
for transducers without weights, as well as OpenFst  by M. Riley, J. Schalkwyk, W. Skut, 
C. Allauzen and M. Mohri~\cite{openfst/2007} for weighted transducers. A structural 
layout of how this compatibility is accomplished in HFST is available in Section~\ref{hfst:structural-layout}.

% 2. Data Formats \ref{hfst:data-formats}
A second set of goals for HFST is to create and collect  
readily available open-source morphologies in order to provide a platform for basic 
high-performing natural language processing tools, to stimulate the production of free 
open-source software for compiling dictionaries, grammars and rules into FSTs, and to 
stimulate the production of language resources (e.g. dictionaries, grammars, rules) to 
be compiled into FSTs. 
HFST offers compatible open-source tools 
for compiling such language descriptions as well as storing them 
in formats that can be exchanged and further processed by the different finite-state libraries 
available in HFST, see Section~\ref{hfst:data-formats}

% 3. SFST Programming Language Compatibility \ref{hfst:sfst-compatibility}
An early adopter of the open-source paradigm was the SFST--Stuttgart Finite-State 
programming language which also has attracted developers of full-scale lexicons for 
various languages, e.g. German, Finnish, Turkish, Italian, etc. The benefits of HFST are 
demonstrated by using the foma library to implement the SFST programming language,
which reduces the compile time to a fraction of the original. For further details on this, 
see Section~\ref{hfst:sfst-compatibility}.

% 4. Xerox Compatibility \ref{hfst:xerox-compatibility}
Over time, the Xerox commercial finite-state environment with compilers like TwolC, 
LexC and Xfst has become popular and many academically developed language 
descriptions are available for these tools. This set of tools is now supported by
HFST as a set of open source tools with an additional tool for composing parallel 
two-level rule sets with a lexicon. This is outlined in Section~\ref{hfst:xerox-compatibility}.

% 5. Runtime and Optimized Lookup \ref{hfst:optimized-lookup}
We provide some insight into our compact and high-speed runtime transducers allowing 
processing speeds of more than 100,000 tokens per second using roughly one percent of 
the size of a corresponding uncompacted file, in Section~\ref{hfst:optimized-lookup}. 

% 6. Application Areas \ref{hfst:application-areas}
Some of the open-source application areas, where HFST is already in use, e.g. 
spell-checking through Voikko for OpenOffice, machine translation preprocessing via Apertium 
and part-of-speech tagging using parallel weighted transducers, are outlined in Section~\ref{hfst:application-areas}.

% 7. Discussion
HFST supports both commercial and open source applications. The tools and libraries of HFST are
compatible with the GNU GPL license, which means that any FST produced with HFST tools will 
remain under the licensing conditions of the input lexicons and rule sets. The HFST runtime
format and runtime library are additionally released under the Apache license. This means that the
HFST tools and the HFST runtime library can be used both for open source and proprietary projects.
A further discussion of the impact of the HFST environment and some of the open-source 
morphologies currently available is provided in Section~\ref{hfst:discussion}.


\section{Structural Layout}\label{hfst:structural-layout}

Finite-state transducer libraries such as SFST~\cite{schmid/2005},
foma~\cite{hulden/2009} and OpenFst~\cite{openfst/2007} all provide
different ways for creating transducers, e.g. foma emulates the Xfst
formalism and SFST contains a compiler for its own regular expression
formalism. The lack of common formalisms makes it difficult to compare
the performance of the different libraries reliably, since they cannot
be used for computing the same tasks. 
 
One of the original goals in the HFST project was to provide a
framework where it is possible to compare the performance of different
finite-state transducer libraries on the same tasks. HFST~2.0 provided
limited support for this by joining SFST and OpenFst under one
interface. Regrettably, adding new libraries was cumbersome in HFST~2.0. 
To remedy this, HFST~3.0 was designed to make it practical to add both
complete and partial implementations of transducer libraries and to use
these for compiling LexC lexicons as well as TwolC, Xfst and SFST grammars.

Related to the goal of comparing performance, is the goal of combining
algorithms from different transducer libraries in one task. This is
now possible in HFST~3.0, where transducers can be converted between
different underlying libraries. Thus it is possible to utilize well
implemented algorithms from a variety of libraries in order to achieve
faster compilation times. It is also possible to test the effect of a
single transducer operation on the total compilation time of a task by
switching between operations from different specialized libraries, i.e. 
it is possible to create specialized data-structures for certain operations 
in a separate library and use them for some special purpose operation 
without the need to re-implement a full set of well-researched 
basic transducer operations.

We first present the general structure of the HFST transducer class library
and then we outline the procedure for adding a new or specialized library.
We then mention the coding principles for exception handling in HFST and
how new libraries can be linked and tested.

\subsection{General Layout of HFST}

Transducers in HFST are objects of the class {\tt HfstTransducer}. The
{\tt HfstTrans\-ducer} class supports all the ordinary transducer
operations like disjunction, composition and automaton determinization. 
{\tt HfstTransducer} encapsulates the different transducer
libraries under the HFST interface, i.e. the same code will work for all
transducer libraries which are part of HFST. E.g. the function {\tt containment} 
in Figure~\ref{containment-figure} works equally well for transducers whose 
underlying implementation is an SFST or a foma transducer.

\begin{figure}
\begin{center}
\caption{Function {\tt containment} computes the language {\tt context center context}, assigns it to center and return a reference to center.}\label{containment-figure}
\begin{verbatim}

  HfstTransducer &containment
  (HfstTransducer &center, const HfstTransducer &context)
  {
    HfstTransducer context_copy(context);
    return center = 
      context_copy.concatenate(center).concatenate(context).minimize();
  }
\end{verbatim}
\end{center}
\end{figure}

Internally {\tt HfstTransducer} objects contain pointers to specific
transducer library implementations. These are wrappers on the actual
transducer libraries. Currently there are four wrappers for three
libraries SFST, OpenFst and foma. The tropical weight semi-ring and log
weight semi-ring in OpenFst have separate wrappers. An {\tt
  HfstTransducer} can be initialized using any of the three transducer
libraries, which are available, and it is possible to convert between
the libraries at runtime.

A wrapper for a library consists of input and output stream classes
for reading and storing binary transducers and a transducer class. The
wrappers encapsulate implementation specific details in the different
transducer libraries thus providing the HFST interface a unified way
to manipulate objects from the different transducer
libraries. E.g. the wrappers for SFST are the classes {\tt
  SfstInputStream}, {\tt SfstOutputStream} and {\tt SfstTransducer}.

Internally each {\tt HfstTransducer} object {\tt t} points to one of
the implementations e.g. {\tt SfstTransducer}. When {\tt t.minimize()}
is called, the {\tt minimize} in {\tt SfstTransducer} gets called. 

In binary operations like {\tt concatenate}, there is risk for a
transducer type mismatch, since the {\tt HfstTransducer} objects
involved may have different types. In case they do, an exception is
thrown. This can be caught and the transducers can be converted to a
common type. The concatenation can then safely take place. 
However, in HFST~3.0, we have made a conscious choice not to convert transducers
automatically in binary operations because this might lose
information, e.g. when converting from weighted to unweighted formats.

\subsection{Adding New Libraries to HFST}

The task of adding a new transducer library {\tt MyTransducerLibrary} to 
HFST can be broken into three subtasks.

\begin{enumerate}
\item Building the wrappers {\tt MyTransducerLibraryInputStream}, {\tt
  MyTransducer\-LibraryOutputStream} and {\tt
  MyTransducerLibraryTransducer}.
\item Adding conversion functions to/from {\tt
  MyTransducerLibraryTransducer} objects from/to the HFST internal
  format {\tt HfstBasicTransducer}.
\item Adding necessary declarations in the master interface files
  {\tt HfstTransdu\-cer.h} and {\tt HfstTransducer.cc} in order to use
  the interface functions properly.
\end{enumerate} 

Unless {\tt MyTransducerLibrary} has an alphabet implementation which
associates string symbols with symbol numbers, such an alphabet also has to
be created.

There is also support in the current HFST interface for transducers
with weighted semi-rings which can not be represented as floating point
numbers, although this may require implementing some new functions.

\subsection{Coding Principles}

Exceptional situations occur in computer programs when the
user does something unexpected or there is a bug in the code.
Examples of user-originating situations in a finite-state library include:

\begin{enumerate} 
\item The user tries to read a binary transducer from a
file that contains a text document or does not exist.
\item A transducer in the AT\&T text format has a typo on one line and the 
line cannot be parsed. 
\item The user calls a function without checking the preconditions,
e.g. tries to extract all paths from a cyclic transducer.
\end{enumerate}

Throwing an exception on such occasions gives the user a possibility
to catch the exception and recover from the situation. In HFST version
2.0, exceptional situations were handled by printing a short message
on the standard error stream and exiting with an error code. In HFST version 3.0, 
exceptions are classes that have a figurative name and contain an 
optional error message.

For the above scenarios, HFST will throw the following exceptions:

\begin{enumerate} 
\item \texttt{NotTransducerStreamException} or 
\texttt{StreamCannotBeReadException} and, in the error message,
the name of the file or stream.
\item \texttt{NotValidAttFormatException} and, in the error message, 
the line that could not be parsed.
\item \texttt{TransducerIsCyclicException}.
\end{enumerate} 

The user could react to the exceptions in the following ways:

\begin{enumerate} 
\item Check that the file exists and contains transducers and try again
with the correct file.
\item Fix the typo in the text format.
\item Call another function that limits the number of paths extracted
from the transducer.
\end{enumerate} 

Exceptions are also used internally in the HFST library for reporting to a 
calling function that something unexpected happened. The calling
function can handle the situation itself or inform the user and suggest
what they should do, which means that the execution of the program 
may continue or terminate gracefully.

An \texttt{HfstFatalException} is thrown when it is unlikely that the user
can handle the exception. The user should instead report the exception
and its circumstances to the HFST developers, because the exception
is essentially a bug that must be fixed. Some assertions are also
used for internal checks. When an assertion fails the user should similarly
report the failure as a bug.

\subsection{Dynamic Linking to Underlying Libraries}
There has been considerable progress in achieving the HSFT goal of acting as a
compatibility layer between different representations of finite-state
transducers and, more importantly, the operations and formalisms (eg.
\verb+LexC+, \verb+TwolC+, \verb+Xfst+, \verb+SFST+) that have been implemented
for them. HFST is now independent of any particular library and requires no
custom extensions to the libraries it uses.

Previously, HFST relied on custom extensions to the libraries it supported,
namely OpenFst and SFST, which made it necessary to statically link them into
\verb+libhfst+. This was obviously also rather restrictive in terms of
new versions and different use cases, e.g. experimenting with local changes to
the underlying libraries.

HFST~3.0 supports conditional compilation of all its elements that provide
interfaces to underlying libraries, and dynamic linking is done to whichever
libraries the user configures HFST~3.0 to use.

Due to the recent improvements, HFST~3.0 can also be built without any external
libraries in which case the code supports only the operations for a simple internal
representation and optimized-lookup, cf. Section~\ref{hfst:optimized-lookup},
i.e. building from text representation and fast lookup. If the user's own library is included, 
the code will also support compilation into the optimized-lookup format.


\section{Data Formats}\label{hfst:data-formats}
Each library is also expected to be able to handle its own external binary data format.
For handling the external binary data of a specific library correctly with the HFST command line tools, 
the external binary data files are prepended with a header. Below, we outline the structure 
of this header.

Each of the underlying libraries of HFST is expected to have its own internal data format. 
As mentioned in the previous section, when adding a new library, conversion functions 
to and from the library-specific internal format need to be provided. We outline the procedure
for this which normally is linear in time.

In addition, we may need to create transducers which deal with symbols that have
not yet been specified in their alphabet. For this reason, we also need functions for harmonizing 
the alphabets between two or more transducers of the same format. We give the
preconditions for this operation and refer the interested reader to the literature.

\subsection{Transducer binary format}
An HFST transducer in binary format consists of an HFST header followed by the
back-end implementation in binary 
format\footnote{\url{https://kitwiki.csc.fi/twiki/bin/view/KitWiki/HfstTransducerHeader}}. 
In version 3.0, the header
format is less error-prone than in the previous versions as it gives more 
information both for users of HFST, when seen on a screen or in a
text editor in binary format, and for the HFST library itself.
 
The current header format is somewhat similar to foma where pieces of 
information are separated by newline characters to make them more
readable. In HFST version 2.0, we represented the properties of a
transducer in a two-byte bit vector akin to the OpenFst header
format. The type of the transducer and the existence of an optional
alphabet in the transducer were also encoded with two characters in the beginning of the
binary transducer which required familiarity with the specifications to interpret.

The beginning of an HFST version 3.0 header contains an identifier
'HFST', a separating zero byte, two bytes signifying the length of
the rest of the header in bytes and a separating zero byte. The rest
of the header contains pairs of attributes and their values. The pairs
are separated by newlines, and attributes and values by zero bytes. 
An HFST version 3.0 header must contain at least the attributes
'version', 'type' and 'name' (in that order) and their values. 
Additional attributes can follow after these obligatory ones. 
For instance, we can include information on the minimality or
cyclicity of a transducer even if the back-end implementation does not
store these properties in its binary format, e.g. SFST. 

The new header format makes it easier to react to unexpected
situations and inform the user, if necessary. When we read an HFST
binary transducer, we first see whether the identifier 'HFST' is
found. If not, we know that the user has given an incorrect
file type and can throw an appropriate exception. 
Next we recognize the implementation type of the transducer. 
If the back-end transducer library is not linked to HFST, we can handle
the situation by throwing another exception. 
Then we recognize the version of the header in order to process 
the rest of the header and the back-end implementation correctly. 
If the user has requested a verbose mode for a tool that is reading the
transducer, it is also possible to print the name of each transducer 
before or after reading it.

\subsection{Conversion Between Different Back-End Formats}
In HFST version 3.0, the conversion between different back-end formats, i.e. 
SFST, foma or OpenFst with tropical and logarithmic semi-ring,  is
carried out through the prorpietary HFST transducer format, \texttt{HfstTransitionGraph}. 
The HFST internal format is a simple transition graph data type that consists 
of states (unsigned integers) and transitions between those 
states\footnote{\url{http://hfst.sourceforge.net/hfst3/index.html}}.
 
We have chosen to implement \texttt{HfstTransitionGraph} for two
reasons. Firstly, it serves as an intermediate transducer format in
conversions, thus reducing the number of conversion functions to
$2 \times N$ from $N \times (N - 1)$, where $N$ is the number of different
transducer back-end formats. Secondly, it is easy to implement functions for 
\texttt{HfstTransitionGraph} that allow the user to construct transducers from 
scratch and iterate through their states and transitions. Implementing
such features for an existing transducer library, can sometimes require modifications
of the library if the library is designed to be used on a higher level of abstraction, 
e.g. in SFST and foma, the functions that operate on states and 
transitions were protected and not well-documented.

\texttt{HfstTransitionGraph} is a class template with template parameters 
\texttt{T} and \texttt{W}. \texttt{T} defines the type of transition data
that a transition uses and \texttt{W} the weight type that is used 
in transitions and final states. 
\texttt{HfstTransitionGraph} contains two maps. One maps each state to a set of
the state's transitions that are of the type \texttt{class HfstTransition\textless class T\textgreater}. 
The other maps each final state to its final weight that is of type class \texttt{W}). 
Class \texttt{T} must use the weight type \texttt{W}. 
A state's transition \texttt{class HfstTransition\textless class T\textgreater} 
contains a target state and a transition data field that is of type class 
\texttt{T}.

Actually, \texttt{HfstTransitionGraph} is not a transducer but a more
generalized transition graph that can contain many kinds of data in
its transitions. Currently, the HFST library offers the specializations 
\texttt{HfstBasicTransducer} and \texttt{HfstBasicTransition} 
for \texttt{HfstTransitionGraph} and \texttt{HfstTransition}.
These specializations are designed for weighted 
transducers. The weight class \texttt{W} is a float and the transition data
class \texttt{T} contains an input string, an output string and a weight of
type float. The specializations \texttt{HfstBasicTransducer} and
\texttt{HfstBasicTransition} are used when converting between different
transducer back-end formats.

The class template \texttt{HfstTransitionGraph} is designed so that it can
easily be extended to different kinds of transition data types. E.g., 
if HFST tools are used in text-to-speech or speech-to-text
conversion, the weights may be more complex and the symbol type of the transitions 
will probably be something else than strings.

\subsection{Alphabet}
The alphabet of a transducer means all symbols (strings) that are
known to that transducer. The alphabet includes all symbols that occur 
or have occurred in the transitions of the transducer unless explicitly
removed from the alphabet. 
If we apply a binary operation (e.g. disjunction or composition) on 
transducers $A$ and $B$, the resulting transducer's alphabet will include
all symbols that were in the alphabets of $A$ and $B$. 

In HFST version 2, alphabets were designed to be external to the transducer,
and the interface did not offer any way convenient way for the user to access a 
transducer's internal representation of the alphabet. It was up to the back-end 
implementation to take care of the alphabet of an individual transducer. In SFST 
the transducers always have an explicit alphabet, but in OpenFst their use is optional.
 
In HFST version 3.0, we need to be aware of transducer-specific alphabets 
because two new special symbols are included, \texttt{unknown} and \texttt{identity}. 
These special symbols are a part of the Xerox Finite-State Tool (Xfst) 
formalism~\cite{beesley/2003} and they are also implemented in
foma~\cite{hulden/2009}. The \texttt{unknown} and \texttt{identity} symbols 
are useful when we want to refer to all symbols that are not currently 
known to a transducer but which the transducer can later become aware of. 

Supporting \texttt{unknown} and \texttt{identity} symbols in all HFST back-end
implementations has enabled us to provide an Xfst compiler that can be used
with all back-end implementations. 
In this way, we can offer users of HFST a new 
formalism for regular expressions in addition to the one available in SFST. 
As the extension is already implemented in foma, we only needed to 
consider SFST and OpenFst in HFST~3.0.

Aside from keeping track of all symbols known to an individual transducer, we also
have to expand each transition involving \texttt{unknown} and 
\texttt{identity} symbols into a set of transitions every time we apply 
a binary operation on two transducers. 
This is because the transducer becomes aware of new symbols that are
no longer unknown and thus no longer included in the \texttt{unknown} or 
\texttt{identity} symbols.
Fortunately, this expansion can be done before the operation itself 
(and for composition before and after the operation itself), i.e. it is  
not necessary to make changes in the operations of the back-end transducer
libraries. 
The library operations can and will handle the special symbols just
like any ordinary symbols. 

First we iterate through the alphabets of both transducers and find
out which symbols in the alphabet of one transducer are not found in
the alphabet of the other transducer and vice versa. 
Then we add beside each transition involving the \texttt{unknown} or 
\texttt{identity} symbols a set of transitions, where \texttt{unknown} 
and \texttt{identity} symbols are
replaced with all symbols that the transducer just became aware of. 
For more information on how to expand special symbols, 
see~\cite{hulden/2009} and~\cite{beesley/2003}.

It is also possible to switch off the handling of special symbols if we know
for sure that they are not used in the transducers. 
In this way, we can optimize performance for instance for the tool 
{\tt hfst-calculate} that processes the SFST programming language 
formalism which does not support the unknown or identity symbols.


\section{SFST Programming Language Compatibility}\label{hfst:sfst-compatibility}
The performance of HFST has improved from version 2.0 to 3.0. 
We compiled two finite-state morphologies in the SFST programming language format
with HFST versions 2.0 and 3.0. 
The morphologies were OMorFi~\cite{pirinen/2008} for Finnish and 
Morphisto~\cite{zielinski/2009} for German.
In Table~\ref{tab:compilation_times} are the compilation times 
for both morphologies with 
different back-end implementations with both versions of HFST. 
Note that the foma implementation was not available in version 2.0.

\begin{table}
\centering
  \begin{tabular}{ c | c | c | c }
  \multicolumn{4}{c}{Compilation times} \\ \hline
  Back-End                  & version & Finnish & German \\ \hline
  \multirow{2}{*}{SFST}    & 2.0     & 25:16   & 107:47 \\
                           & 3.0     & 5:02    & 6:39 \\ \hline
  \multirow{2}{*}{OpenFst} & 2.0     & 7:54    & 6:23 \\
                           & 3.0     & 6:51    & 6:28 \\ \hline
  \multirow{2}{*}{foma}    & 2.0     & -       & - \\
                           & 3.0     & 1:49    & 1:29 \\ 
  \end{tabular}
  \vskip.2cm
  \caption{Compilation times for Finnish and German morphologies with
    HFST. The times are expressed in minutes and seconds.}
  \label{tab:compilation_times}
\end{table}
%% FMT:  Perhaps mention how the numbers were calculated, are they averages of several runs?
%% i.e. is the 5 second difference significant?

We can clearly see that the compilation time has improved dramatically
for the SFST implementation.
This is mainly because the new version of SFST, 1.4.2, uses Hopcroft's
minimization algorithm~\cite{hopcroft/1971} instead of 
Brzozowski's~\cite{brzozowski/1964}. 
We noticed that the Brozowski minimization algorithm hampered the performance
already when we were testing HFST version 2.0; 
OpenFst was clearly faster because it used the Hopcroft algorithm. 
Based on this observation, Helmut Schmid improved SFST by 
writing a minimization function using the Hopcroft algorithm.

When comparing the compilation times for OpenFst, we see that the
Finnish morphology compiles faster but the German one slightly slower
on HFST version 3.0 than on version 2.0. This is because there are two
main factors that contribute to the difference in performance. Firstly, we
are currently using OpenFst version 1.2.7 that is faster than the
previous versions. Secondly, in HFST version 3.0 we no longer use a
global number-to-symbol encoding for all transducers during one
session. Every time we perform a binary operation on two transducers,
we harmonize the encodings of the transducers. Nevertheless, 
it seems that the newer, more efficient version of 
OpenFst mostly compensates for this additional effort caused by harmonization. 

We did not have the foma implementation of the SFST programming language 
available in HFST version 2.0, but it is evident that it is much faster than the other
implementations in either version of HFST. Foma does not use
a global symbol-to-number encoding in its transducers either, but it still
performs well. This is evidence that symbol harmonization is not a big
factor in the compilation times of morphologies. 


\section{Xerox Compatibility}\label{hfst:xerox-compatibility}
Among the goals of the HFST framework has always been to retain legacy support
for the Xerox line of tools for building finite-state
morphologies~\cite{beesley/2003}. Optimally, the tools should be familiar
to end-users converting from the Xerox tools. For this purpose we have aimed to
create clones of the most important Xerox tools as accurately as possible.
Previous open-source implementations of Xerox tool clones have included
LexC and TwolC~\cite{linden/2009/sfcm} and LexC and Xfst~\cite{hulden/2009}; in
HFST~3.0 we have combined these contributions into one uniform package capable of
handling the full line of Xerox tools for morphology. For the most part,
end-users will require no other familiarization than changing program names in
order to start using HFST tools for their Xerox-style language description
needs.

The implementation of the Xfst scripting language makes heavy
use of the new foma back-end, which already had a good coverage of Xfst
features. The only additions in HFST are the ones required for
interoperability between other back-ends and HFST internals. Particular
care was taken not to duplicate the work already present in foma and its
tools. Similarly HFST's LexC parsing engine was
replaced by the faster LexC parser in foma, again with HFST interoperability
tweaks straddling the gaps.

For practical examples of specific previously implemented Xerox style finite-state
language descriptions we provide a wiki-based web page\footnote{\url{https://kitwiki.csc.fi/twiki/bin/view/KitWiki/HfstExamples}}. Another repository of
such language descriptions is located at the University of Tromsø's subversion
repository\footnote{\url{http://divvun.no/doc/infra/anonymous-svn.html}}.
Of these, the morphologies for the Sámi languages and Greenlandic are 
regularly used in regression and stress tests of the HFST tools.

For specific functionalities, the Xerox tools perform various kinds of special
processing of finite-state transducers beyond the range of standard finite-state
algorithms. A prominent example of this is the handling of special symbols such
as flag diacritics~\cite{beesley/1998}, which would require support from
the underlying libraries for many finite-state operations to work as they do in
Xerox tools when using \texttt{flag-is-epsilon} and \texttt{obey-flags}
settings. HFST tools provide support for such options, and provide
fall-back processing where back-end libraries lack support for the
required operations. Fall-back support commonly involves converting the
back-end library internal transducers to the HFST internal format, calculating the operation
and converting the transducer back to the back-end library format. 

\subsection{Intersecting Composition}
Intersecting composition is used for applying a grammar of two-level
rules to a two-level lexicon. The result of the operation
is, e.g. a morphological analyzer mapping word forms to
analyses. Compiling the analyzer using conventional methods requires
computing the intersection of the rule transducers. This may lead to a
prohibitively large intermediate result. Intersecting composition
avoids computing the entire intersection of the rules thus reducing
both memory and time requirement. The operation was introduced by
Karttunen~\cite{Karttunen/1994} and later extended to weighted
transducer in HFST~2.0 by Silfverberg and Lind\'{e}n~\cite{silfverberg/2009/2}.

The intersecting composition operation was implemented in HFST~2.0, but
we used techniques adopted from OpenFst~\cite{openfst/2007} to improve
the implementation, and the current implementation is significantly
faster than the old one. The current implementation computes a lazy
pairwise intersection of the rule transducers. The lexicon can be
composed with this intersection using a standard composition
algorithm.

\subsubsection{Previous Implementation.}
The implementation of intersecting composition in HFST~2.0 can be
characterized as the composition of the lexicon transducer $L$ with a
structure $P$ containing all rule transducers. For simplicity,
we assume that the lexicon and rule transducers are deterministic.

Outwards the structure $P$ resembles an ordinary transducer with
states and transitions. The states of $P$ internally correspond to
vectors of rule transducer states, which we call state
configurations. The vectors have as many indexes as there are rules
and each rule corresponds to a unique index, where its state is
stored. E.g. the start state of $P$ corresponds to the vector
containing the start states of the rules.

Initially only the start state of $P$ is computed. More states in $P$
are computed according to the transitions in $L$. E.g. the lexicon $L$
might have a transition with output-symbol {\tt a} in its initial
state. In order to compute the intersecting composition, it would be
necessary to create transitions and corresponding target states in $P$
for all symbol pairs {\tt a:b}, where each of the rules has
transitions from its initial state with symbol pair {\tt
  a:b}. Outwards $P$ would have one target state $t$ for the
transition with pair {\tt a:b} from its initial state. Internally $t$
would correspond to a configuration of the target states of the
transitions with symbol pair {\tt a:b} in each of the individual rules.

There is no caching of transitions in the states of $P$. Thus the
transitions in states have to be recomputed every time the algorithm
visits a given configuration of rule states. This requires more work
than if the transitions were cached, since there usually exist state
configuration which are very frequently visited.

Even if the transitions in states were cached, this implementation
would still be suboptimal, since a new state configuration always
requires re-examining the transitions in all rules. This is true even
if only one rule state differs from a previous configuration.

\subsubsection{Current Implementation.}
We note that phonological two-level rules usually track sound changes in fairly
specific contexts. This means that when composing two-level rule
transducers with a lexicon, the rules will occupy a limited state set
during the majority of the time of the composition. The current
implementation of intersecting composition capitalizes on this
property.

Instead of a parallel lazy intersection like we used in HFST~2.0, we
recursively build an intersection of the rules by intersecting them
lazily pairwise. The first and second rule are intersected, the result
is intersected with the third rule and so on. The HFST~3.0 implementation 
caches the transitions of a given state pair, so there
is no need to recompute them when the state is revisited.

This leads to significant improvement in performance. The improvement
results from caching transitions, but it is improved by the fact that
computing the transitions in a previously unseen state configuration
does not require recomputing the transitions in all of the
rules. E.g. if rule number $n$ moves to a new state with symbol pair
{\tt a:b}, but the rest of the rules remain in a familiar state
configuration, we only need to recompute the transitions of the rules
having greater index than $n$. This derives from the fact that we have
already cached the target state of the subset of rules $1$ to $n-1$ in
their lazy intersection structure.

Like the old implementation, the current implementation of
intersecting composition is equivalent with Xerox style flag diacritics
and {\tt identity} symbols.

\begin{table}[htb!]
\begin{center}
\begin{tabular}{l|c|c}
Language  & HFST~2.0 & HFST~3.0\\
\hline
North S\'{a}mi  & 364.2 s & 63.4s \\
\hline
Finnish   &  4.1s  & 2.6s   
\end{tabular}
\vskip.2cm
\caption{Runtimes for intersecting composition of the Finnish and Northern S\'{a}mi morphological analyzers in HFST~2.0 and HFST~3.0.}
\end{center}
\end{table}


\section{Runtime and Optimized Lookup}\label{hfst:optimized-lookup}
\emph{Optimized-lookup} is an HFST-specific binary format for finite-state
transducers providing fast lookup. First documented in \cite{silfverberg/2009},
its implementation has evolved somewhat to meet the requirements of specific
applications. The format has also found new application in on-the-fly
operations in transducers, e.g. composing lookup for spell-checking and
correction,  see Section~\ref{spellcheck}).

\subsection{Implementation and Integration in HFST~2.0 and HFST~3.0}
Optimized-lookup was supported in HFST~2.0 by the standalone utilities\\
\verb+hfst-lookup-optimize+ (compilation) and
\verb+hfst-optimized-lookup+ (non-token\-izing lookup). This was partly
in service of the goal of giving optimized-lookup the widest possible range
of uses; \verb+hfst-optimized-lookup+ was released under the
Apache License~\cite{apache-license}, whereas HFST~2.0 proper was released
under the potentially more restrictive GNU Lesser Public
License~\cite{lgpl-license}.

Provision was later made for both unweighted and weighted (with log weights)
transducers, and flag diacritics~\cite{beesley/2003}, see also Section~\ref{flag-diacritics}.

\subsubsection{Other Implementations and Applications.}
Demonstrations of the lookup facility were also
produced in Java and Python, two popular and accessible programming languages,
in the hope of facilitating and spreading use of the format. This effort
bore fruit in the incorporation of the Java code in a project for anonymizing
identities in legal documents at the Aalto University in Helsinki.

The format saw additional uses and implementations over the course of 
furthering research goals and maintaining HFST~2.0. Hyphenators and spell
checking transducers were primarily used in this format for its speed, and
in 2010 a Google Summer of Code\footnote{\url{http://code.google.com/soc/}} project by Brian Croom produced
\verb+hfst-proc+, a tokenizing lookup application for optimized-lookup which
was put to use in various text stream processing scripts (e.g.
\verb+-analyze+ for analysis, \verb+-generate+ for generation and
\verb+-hyphenate+ for hyphenation).

\subsubsection{HFST~3.0.}
Originally compilation to the format was only possible from the SFST and OpenFst
formats, and as uses and applications proliferated, it became desirable to
provide some API access to optimized-lookup in the HFST library. In HFST~3.0
this has been accomplished by implementing compilation from the HFST internal
transducer format, allowing for a great degree of integration with
HFST~3.0 supported tools.

\subsection{Index Table Compaction}
The crucial idea behind optimized-lookup is Liang compaction, as described in
\cite{silfverberg/2009}. It allows for the representation of a transducer as
a lookup table, with entries for each symbol in the alphabet for each state
in the transducer, without growing to the prohibitive sizes such a design
would imply, i.e. a multiple of the number of states and the number of symbols for
the state indexing table alone. Liang in his PhD thesis on hyphenation
\cite{liang/1983} did not specify a generalized compaction scheme, only the
requirements for its correctness. In realistic transducers, finding the optimal
compaction is in any case computationally infeasible, and consequently some approaches to
producing a ``good enough'' compaction have been attempted.

For the purposes of this article, the task of compacting the index table may
be summarized as the following. Given $N$ arrays $s$ of length $L$, the
entries of which are $0$ or $1$, calculate a list of starting indexes
$I_1, I_2 \ldots I_N$ such that a result array with cells
\begin{equation}
R_i = \displaystyle\sum\limits_{p, q} s_p(q) [I_p + q = i]
\end{equation} will also have
entries $0$ or $1$. An optimally compacted index table corresponds to the
shortest result array $R$.

A simple strategy is to iterate through the arrays in some order, assigning
the lowest possible starting index to each one. For transducers of an
appreciable size this process can become slow, as the result array
will typically have some zeros in practically all its regions, so a large
number of possibilities have to be checked. This problem can be mitigated by
applying a head filter, disregarding the largest region of the result array
$R(1 \ldots r)$ with that region having a density of $1$ entries greater than
some predefined limit. A limit of $1.0$ corresponds to having no filter at all;
in practice limits in the region $0.8 \ldots 0.9$ have proved reasonable.

\begin{table}[htb!]
\begin{center}
\begin{tabular}{c|r}
Head filter limit  & Number of index entries \\
\hline
(No compaction) & 4,541,879 \\
\hline
0.0 & 1,107,321 \\
\hline
0.1 & 408,843 \\
\hline
0.3 & 206,152 \\
\hline
0.5 & 170,789 \\
\hline
0.7 & 155,703 \\
\hline
0.8 & 148,740 \\
\hline
0.9 & 140,795 \\
\hline
1.0 & 135,285 \\
\end{tabular}
\vskip.2cm
\caption{Index table sizes for various values of the head filter limit using
in-order traversal as applied to the Morphalou project's French morphology and
released on the HFST site on 2010-04-14.}
\end{center}
\end{table}

For the order in which the arrays are traversed, ordering the states from
greatest density of $1$ entries to lowest and simply ordering them in
numerical order have been tried. These approaches don't appear to produce
dramatically different results\footnote{Of course, in practice these orderings
will often be similar.}; also in theory, either approach could
produce better results than the other.

Work is ongoing in finding the best practical filter strategies, filter
limits and traversal orders, and potentially different compaction strategies.

A representation of the index table with no compaction would require
$N \times L$ entries. For the sake of comparison, in the case of an analyzing
transducer from OmorFi \footnote{We used the binary released on the HFST site on
2010-10-14.}, a Finnish finite-state morphology, this is
$203,851 \times 155 = 31,596,905$ and would produce a binary of almost 200
megabytes, whereas the compacted index table has 364,980 entries, or about
one percent of the uncompacted form, for a binary of 7 megabytes.

\subsection{Flag Diacritics and Related Optimization Tricks}\label{flag-diacritics}
Support for flag diacritics was added to the \verb+optimized-lookup+ utility
with an eye to efficiency, prompting some refinements to the HFST implementation
of the format itself. Flag diacritics are parsed prior to lookup and during
it they restrict the lookup search tree. This is critical for speed, as the
alternative of calculating all the outputs first and then removing outputs
with conflicting flag diacritics can, in the case of transducers with liberal
use of flags, involve several times the work.

For the purposes of traversing transitions, flag diacritics are essentially
a special case of the epsilon symbol. If the configuration of flags
that have been traversed up to a certain point allow it, each transition
with a flag diacritic is traversed without reading an input symbol. With
this in mind, it was desirable to avoid checking for transitions with each
flag symbol in each state. The optimized-lookup format was therefore amended
to treat flag diacritics as epsilon for purposes of constructing the index
table and to list their transitions out of the normal order, after the epsilon
transitions. Thus it is possible to check only those flag transitions that
are present in a given state.

This allows further reductions in the size of the index table; as
the indexes for flag diacritics are no longer in use, it is possible to reduce
the effective size of the input alphabet by the number of different flag
diacritic operations. These improvements may appear minor, but are not
insignificant in transducers that make substantial use of flag diacritics. In
the previously discussed version of OmorFi, we achieved a reduction of $12.4~\%$
in index table size.


\section{Application Areas}\label{hfst:application-areas}
After the initial release of the HFST platform, it has been used in several end-product
applications. The two most prominent uses are as a part of the rule-based
machine translation platform Apertium\footnote{\url{http://apertium.sf.net}}
and the spell-checking library Voikko\footnote{\url{http://voikko.sf.net}}. Both of
these linguistic applications benefit hugely from the fact that there
were previous language descriptions available written in the Xerox finite-state
morphology formalism, and integrating HFST in these applications gave
developers of those language descriptions a direct conversion path to two
application types that had not been available in the Xerox framework. Since
both applications, as well as the HFST framework, are free/libre open-source
software, the integration of the existing language descriptions in the
projects was possible.

One of the most pressing reasons for extending finite-state support to these
applications is the lack of language support and the lack of a theoretically well-motivated
open-source option for language support for morphologically more complex languages
in the above-mentioned applications. For example in the field of spell-checking the
theoretical upper-bound for hunspell---de facto standard in the open-source market---
is mere 4 affixes. For polysynthetic languages like Greenlandic, it simply is
not possible to precompute enough affix and stem combinations, as has been done with
e.g. Hungarian. The Xerox style finite-state morphology demonstrably supports
at least Hungarian and a wide variety of other morphologically varied
languages~\cite{beesley/2003}.

Another rationale for extending finite-state methods to the application areas
is that the efficiency and expressiveness of finite-state automata is
well-known and has been researched e.g. in \cite{aho/2007}, which makes it a good choice
for various text-processing tasks.

\subsection{Apertium Interoperability---Corpus Processing Tools and I/O Formats}
In Apertium, the finite-state automata are used to perform morphological analysis and generation for
both parsing running text and generating the translations after performing
a mid-shallow rule-based transfer. The HFST software is only one possible morphological analyzer, 
so the crucial part for inclusion was to get the HFST analyzer to work as the competition. 
This included two functions: reliable tokenization based on
the dictionary data and support for Apertium I/O formats. The 
contribution of the corpus processing functionality is contained in a tool
called \texttt{hfst-proc} also included in the HFST toolkit. The name is influenced by
similar corpus processing tools in other toolkits, specifically
\texttt{cg-proc} from VISLCG3\footnote{\url{http://beta.visl.sdu.dk/cg3.html}} and
\texttt{lt-proc} from Apertium itself.

For tokenization the FST-based dictionaries are useful, since the process of
analysis and lookup can both be performed by basic FST traversal. The specific
implementation of analysis and tokenization with a single FST traversal was
implemented as a Google summer of code project, based on previous studies on
the topic~\cite{garrido-alenda/2002}. The basic programming logic of the automata traversal
for the longest match is trivially extended by the processing of flag diacritics
and weights.

The I/O format requirements for the Apertium platform are based on the needs to
translate existing documents containing all kinds of markup and rich text
formats, such as HTML for web pages or MediaWiki codes from Wikipedia. To
achieve this, Apertium uses text encoding and decoding mechanisms and
an interchange format called Apertium stream format, whose input and output
was implemented in the HFST corpus processing tools.

\subsection{Voikko and HFST Based Spell-Checker Formulation}\label{spellcheck}
The application of finite-state morphologies in spell-checking applications
is also based on a new development of finite-state algorithms. The application
framework including HFST spell-checkers is the Voikko library, which provides
spell-checkers for OpenOffice.org/LibreOffice, the GNOME desktop (via
enchant), Mac OS X (via SpellService) and the Mozilla application suite.

The finite-state formulation of a spell-checking system was developed based on
previous research. In this research it has been shown that a finite-state based
natural language description is usable as spell-checker with specialized fuzzy
traversal algorithms~\cite{oflazer/1996,hulden/2009} or by using a special
(weighted) two-tape automaton as an error model and regular finite-state
composition to map misspelled words to their possible
corrections\cite{agata/2002,pirinen/2010/lrec}.
 
The basic finding here is that typical finite-state language descriptions are
usable as spell-checking dictionaries with minor to no modifications.
Furthermore it has been shown that existing non-finite-state spell-checking
dictionaries from hunspell and myspell can be converted into finite-state
form~\cite{pirinen/2010/cla} providing full backwards compatibility for
traditional spell-checking systems.

Furthermore, we have optimized the application of error models when suggesting corrections
by applying a three-way composition with both the dictionary, the error model and the misspelled word in
one operation. This significantly reduces the space and time requirements by leaving many of the impossible
intermediate results uncalculated.

\subsection{Statistical Part-of-speech Tagging using Weighted Finite-State Transducers}
HFST~3.0 has also been applied to part-of-speech tagging the Finnish, Swedish
and English Europarl corpora \cite{silfverberg/2010}. Silfverberg and
Lind\'{e}n implemented first order Hidden Markov Models (HMM) as sets
of parallel weighted finite-state transducers using HFST~3.0 tools. Like
standard HMMs, their tagger used tag sequences. In addition they
included lemmas in their models. Part-of-speech tagging was
accomplished by intersecting composition of a sentence automaton with
the set of tagger transducers.

\section{Discussion}\label{hfst:discussion}
In parallel with HFST, there is the OMor 
project\footnote{http://www.ling.helsinki.fi/kieliteknologia/tutkimus/omor/} for 
creating and/or compiling open-source morphological analyzers for Finnish, Swedish, 
French, German and English. OMorFi is a large-scale Finnish open-source morphological 
transducer lexicon based on words from a dictionary with inflectional codes as well as 
patterns for compounding and derivation. It is available both for the SFST programming 
language as well as a Xerox-style two-level morphology. The Divvun project in Norway 
has created two-level morphological analyzers for Northern and Lule Sámi using 
Xerox tools. In addition, there are several projects having developed SFST lexicons for 
German\footnote{Morphisto~\cite{zielinski/2009}}, 
Italian\footnote{http://dev.sslmit.unibo.it/linguistics/morph-it.php}, Turkish\footnote{TRmorph--\url{http://www.let.rug.nl/~coltekin/trmorph/}}, etc. 
Compiled from Hunspell sources and large corpora, close to a hundred HFST 
spellers~\cite{pirinen/2010/cla} with an improved error correction mechanism 
already exist. The effort to collect existing morphological descriptions
for various languages is on-going.

A concrete outcome of the current HFST environment is that it is now possible to take 
a lexicon developed with e.g. Xfst or Hunspell and weight it with material from a specialized corpus 
in order to create a tailored speller for a given domain. This can all be accomplished in the course 
of an afternoon after which we can start using it in e.g. OpenOffice.

As future work, we are investigating how to extend the morphological analyzers 
into finite-state implementations of constraint 
grammars\footnote{\url{http://beta.visl.sdu.dk/cg3.html}} and other 
dependency-related tagger and grammar formalisms using both statistical
and rule-based approaches with weighted finite-state transducers. 

%
% trmorph
% Çağrı Çöltekin (2010). A Freely Available Morphological Analyzer for Turkish 
% In Proceedings of the 7th International Conference on Language Resources and Evaluation (LREC2010), 
% Valletta, Malta, May 2010.
%
% http://dev.sslmit.unibo.it/linguistics/morph-it.php
% Eros Zanchetta and Marco Baroni (2005) Morph-it! A free corpus-based morphological resource for the Italian language,
% In Proceedings of Corpus Linguistics 2005, University of Birmingham, Birmingham, UK.
%

\section{Conclusion}\label{hfst:conclusion}
In this article, we have described the structural layout of HFST--the Helsinki Finite-State Technology library
and how it connects important existing finite-state libraries and how it can accommodate additional software
libraries for finite-state algorithms and applications. Especially facilitating exchange of data between the different
finite-state implementations is important for processing and enhancing language descriptions created with different
tools and formalisms. In this article, we especially focused on two finite-state programming language environments, 
i.e. the SFST Programming Language and the Xerox tools for creating morphological descriptions and showed
that the HFST toolkit can cover a wider range of languages and applications compared with the original finite-state environments. We have also demonstrated that the cross-usage of finite-state programming language 
front-ends and finite-state library back-ends can provide significant reductions in processing time.

\section*{Acknowledgments}

\bibliographystyle{splncs03}
\bibliography{sfcm-2011}

\end{document}
% vim: set spell

\documentclass[t,12pt,pdftex]{beamer}
\usepackage{helvet}
\usepackage{times}
\usepackage{courier}

\usepackage[T1]{fontenc}
\usepackage[english]{babel}

\usepackage{amssymb}
\usepackage{amsmath}
\usepackage{amsfonts}
\usepackage{graphicx}
\usepackage{color}
\usepackage{url}
\usepackage{textpos}
\usepackage{xspace}
\usepackage{array}

\graphicspath{{./fig/}}

% theme options: hy/ml/hum, rovio/sinetti, hiit
% default: hy,rovio

%\usetheme[hy]{HY}
%\usetheme[hy,sinetti]{HY}
\usetheme[hum,rovio]{HY}
%\usetheme[ml,rovio]{HY}
%\usetheme[ml,rovio,hiit]{HY}


\title{Building Real-World Finite-State Spell-Checkers With HFST\\
in FSMNLP 2012 at Donostia}
%[Tut: HFST Spellers]
\author{Tommi A Pirinen \scriptsize \guilsinglleft{}tommi.pirinen@helsinki.fi\guilsinglright{}}
\institute{University of Helsinki\\Department of Modern Languages}
\date{\today}

\begin{document}

\selectlanguage{english}

\HyTitle

\begin{frame}
    \frametitle{Outline}
    \tableofcontents
\end{frame}


\AtBeginSection[]
{
  \begin{frame}<beamer>
    \frametitle{Outline}
    \tableofcontents[currentsection]
  \end{frame}
}

\section{Introduction}

\begin{frame}
    \frametitle{Finite-State Spell-Checking}
    
    \begin{itemize}
        \item A simple task: go through all words in text to see if they
            belong to language LL, if not, modify them with relation
            EE to fit into language LL
        \item In Finite-State world language model LL is any (weighted) single
            tape finite-state automaton recognising the words of the
            language
        \item The error model EE is any (weighted) two-tape automaton, that
            describes spelling errors, i.e. mapping from misspelt word into
            the correct one
        \item In this tutorial we build simple language model and piece
            together error model from different error types
    \end{itemize}
\end{frame}

\begin{frame}
    \frametitle{With HFST Tools (but mostly generic FST algebra)}
    
    \begin{itemize}
        \item The tools we use to build the automata are from HFST project
            \url{http://hfst.sf.net}
        \item but most what I show here is finite-state algebra, which your
            favorite fst tools should support as well
        \item The chain of libraries to get HFST automata spellers to any
            desktop application is HFST 
            ospell$\rightarrow$voikko$\rightarrow$enchant (with few
            exceptional application not supporting this). 
            \url{https://kitwiki.csc.fi/twiki/bin/view/KitWiki/HfstUseAsSpellChecker}
    \end{itemize}
\end{frame}

\section{Languege Models}

\begin{frame}
    \frametitle{The very simplest of language models---list of words}
    
\end{frame}

\begin{frame}
    \frametitle{Other language models}
\end{frame}

\section{Error Models}

\begin{frame}
    \frametitle{Basic component 1: Run of correctly spelled characters}
\end{frame}

\begin{frame}
    \frametitle{Basic component 2: Typing error}
\end{frame}

\begin{frame}
    \frametitle{1 + 2 = 3: Edit Distance Automata}
\end{frame}

\begin{frame}
    \frametitle{Basic component 4: Confusion set of words}
\end{frame}

\begin{frame}
    \frametitle{Basic Component 5: Common Phonemic Mispellings (simple)}
\end{frame}

\begin{frame}
    \frametitle{Component 6: Misspelling in context}
\end{frame}

\begin{frame}
    \frametitle{Finally: hacking all these together}

\end{frame}

\section{Misc. practicalities}

\begin{frame}
    \frametitle{Metadata}

\end{frame}

\begin{frame}
    \frametitle{Zipping and Installation Location}

\end{frame}

\end{document}

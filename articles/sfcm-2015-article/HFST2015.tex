\documentclass{llncs}

\usepackage{llncsdoc}

%% PDFLaTeX
\usepackage[T1]{fontenc}
\usepackage[utf8]{inputenc}
\usepackage{lmodern}
\usepackage{textcomp}      % for ° symbol
\usepackage{multirow}
\usepackage{caption}
\usepackage{url}
\usepackage{tabularx}
\usepackage{tikz}
\usetikzlibrary{automata,positioning}
\usepackage{framed}

%% XeLaTeX
% \usepackage{fontspec}
% \usepackage{xunicode}
% \usepackage{xltxtra}

\usepackage{expex}

%
\begin{document}
%
\title{Semantic tagging using HFST}
%
\author{N.N. and N.N.}

 \institute{xx\\
 yy\\
 zz\\
 ww\\
 \email{\{n.n., n.n.\}@xxx.yy}}

\maketitle

% Removed the manual bibstyle in favor of splncs03.bst,
% if it's required fetch it from svn history

\begin{abstract}
%Krister
\end{abstract}

\section*{Introduction}
% Krister (1 p.)

\section{Tokenization using {\tt hfst-pmatch}}\label{sec:tokenization}
% Sam (4 p.)
Tokenization is a necessary first step in most text-based NLP tasks. For some
languages (eg. English) it is often considered to be a mechanical
preprocessing task without linguistic importance, and for others (eg. Chinese)
it is a subtle task given a different name (``segmentation'').

However, even in languages that generally insert spaces between words, there
are issues that influence the quality or feasibility of tools down the
pipeline. We may, for example, want to be able to identify multiword units,
identify compound words and mark their internal boundaries, control various
dimensions of normalisation, or produce possible part-of-speech tags or
deeper morphological analyses.

We describe a general approach to these issues based on morphological
transducers, regular expressions and the pattern matching operation
\verb+pmatch+~\cite{pmatchcite}.

\subsection{Tokenizing with a Dictionary}

\subsubsection{Preserving the Parts of a Multiword Unit}

\subsection{Tokenization Rules as an OOV Fallback}

\subsection{Analysis Cohorts}

\subsection{Chunking}

\subsection{Incorporating other Linguistic Resources}

\section{Morphological Tagging using {\tt hfst-finnpos}}\label{sec:morph-tagging}
% Miikka (4 p.)

Finnpos \cite{silfverberg2015} is a morphological tagger toolkit based
on the Conditional Random Field framework. It is especially geared
toward morphologically rich languages.  It utilizes several
optimizations to ensure fast estimation and inference even with large
label sets. Additionally, it uses sub label dependencies in
unstructured and structured features for counteracting data sparsity
and modeling grammatical phenomena such as case and number congruence.

Besides optimizations and sub-label dependencies
\cite{silfverberg2014}, Finnpos also provides a flexible way of
integrating a morphological analyzer in the tagging
process. Morphological analyses can be use both as features during
estimation and inference and for constraining the set of possible
morphological labels for word forms. Additionally, the analyzer is
used in lemmatization.

For words not recognized by the morphological analyzer, Finnpos
includes a data driven lemmatizer which is based on the averaged
perceptron classifier..

\section{Semantic Tagging using {\tt hfst-pmatch}}\label{sec:sem-tagging}
% Sam (4 p.)

\section{Distortion Filtering with Weighted Regular Expressions}
% Sam & Miikka (3 p.)

\section{Background}\label{sec:background}
% Erik (2 p.)

\section{Discussion and Conclusion}\label{sec:discussion}
% Krister (1 p.)

\bibliographystyle{splncs03}
\bibliography{sfcm-2015}

\end{document}
% vim: set spell:

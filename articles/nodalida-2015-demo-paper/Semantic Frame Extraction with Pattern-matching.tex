\documentclass[11pt]{article}
\usepackage{nodalida2015}
\usepackage{times}
\usepackage{mathptmx}
%\usepackage{txfonts}
\usepackage{url}
\usepackage{latexsym}
\usepackage[T1]{fontenc}
\usepackage[utf8]{inputenc}
\special{papersize=210mm,297mm} % to avoid having to use "-t a4" with dvips 
%\setlength\titlebox{6.5cm}  % You can expand the title box if you really have to

\title{Extracting Semantic Frames with Pattern-matching}

\author{Sam Hardwick \\
  University of Helsinki \\
  {\tt sam.hardwick@iki.fi} \\\And
  Miikka Silfverberg \\
  University of Helsinki \\
  {\tt miikka.silfverberg@iki.fi}\\\And
  Krister Lind{é}n \\
  University of Helsinki \\
  {\tt krister.linden@helsinki.fi}}

\date{}

\begin{document}
\maketitle
\begin{abstract}
  We demonstrate the use of \verb+hfst-pmatch+~\newcite{pmatch}, a development of Xerox
  \verb+fst+~\newcite{xerox-pmatch}, for developing a semantic frame extractor. We
  select a FrameNet~\newcite{framenet} frame and write shallowly syntactic pattern-matching rules
  based on part-of-speech information and morphology from either a morphological automaton or
  tagged text.
\end{abstract}

\section{Introduction}
\verb+pmatch+ is a pattern-matching operation based on regular expressions. It
allows embedded references to other expressions (recursive transition networks
or RTNs), including circular or self-references, which allows matching
context-free grammars. The matched patterns may be efficiently tagged, extracted
and modified by the rules.

Substantial named-entity recognisers (NERs) have been developed in
\verb+pmatch+ for Swedish and Finnish. Here we demonstrate a bottom-up approach
to using it to identify the frame ``Size'' in FrameNet.

\section{A Semantic Frame}

A semantic frame is a description of a \emph{type} of event, relation or entity
and related participants~\newcite{semantic-frame}. For example, in Framenet,
a database of semantic frames,
the description of an \verb+Entity+ in terms of physical space occupied by it is
an instance of the semantic frame \verb+Size+. The frame is characterised by
a lexical unit (LU), which is a word (in this case an adjective)
such as ``big'' or ``tiny'', descriptive of the size of the \verb+Entity+.
Apart from \verb+Entity+, which is a core or compulsory element, the
frame may identify a \verb+Degree+ to which the \verb+Entity+ deviates
from the norm (``a \textbf{really} big dog'') and a \verb+Standard+ to
which it is compared (``tall \textbf{for a jockey}'').

\section{A Rule}

A simple and common syntactic realisation of the \verb+Size+ frame is a single
noun phrase containing one of the LUs, such as
``the big dog that ran away''. Here we'd like to identify ``big'' as \verb+LU+,
``dog'' as \verb+Entity+ and the combination as \verb+Size+.
Our first rule for identifying this type of construction might be

\begin{verbatim}
Define Size1 LC(W) Define(LU EndTag(LU))
  W Define(Noun EndTag(Entity)) RC(W);
Define TOP Size1 EndTag(Size);  
\end{verbatim}

\verb+TOP+ is the root of the pattern matcher, and must be present.
\verb+W+ is a word boundary, such as a space character or the beginning or end of
input. \verb+LC+ and \verb+RC+ are special operations for checking that at both left and right
sides of the rule there is a word boundary without including it in the match itself.
The \verb+Define()+s inside other definitions are anonymous rules that delimit the effect of
\verb+EndTag()+, which assigns a tag to a pattern.

We can verify that this extracts instances of our desired pattern by compiling
these rules with \verb+hfst-pmatch2fst+ and running the compiled result with
\verb+hfst-pmatch --locate+, which gives the locations (which we omit here to save space) of matches and omits
nonmatching text. Here we have inputted the text of the King James Bible from
Project Gutenberg~\newcite{projectgutenberg} and added some extra characters on both sides for
a concordance-like effect:

\begin{verbatim}
...
let your little ones also go|<Size>
all be a great cry througho|<Size>
re was a great cry in Egypt|<Size>
saw that great work which th|<Size>
re lay a small round thing, a|<Size>
...
\end{verbatim}

\subsection{Incorporating Semantic Information}
Size is very metaphorical, so we can get a list of concrete nouns from WordNet
or else laboriously iterate and collect metaphorically used nouns ourselves...

\subsection{Incorporating Part-of-speech Information}

We have hitherto used named rules for matching word classes, like \verb+Noun+,
without specifying how they are written. Even our collection of LUs might need
some closer attention -- for example ``little'' could be an adverb.

Considering that in writing our
rules we are effectively doing shallow syntactic parsing, even a very simple
way to identify parts of speech may suffice: a morphological dictionary.
For example, a finite-state transducer representing English morphology may be
used to define the class of common nouns as follows:

\begin{verbatim}
! The file we want to read
Define English @bin"english.hfst";
! We compose it with a noun filter
! and extract the input side
Define Noun English .o.
 [?+ "<NN1>" | "<NN2>"].u;
! (NN1 is singular, NN2 plural)
\end{verbatim}

If we have the use of a part-of-speech tagger, we may write our rules to act
on its output:

\begin{verbatim}
Define Noun LC(W) [\W]+
 ["<NN1>"|"<NN2>"] RC(W);
\end{verbatim}

\section{Increasing Coverage}
Identify common cases we missed by finding LUs that aren't matched by
anything and write more rules...

\section{Elaborations}

More sophisticated parsing with embedded structures...

% If you use BibTeX with a bib file named eacl2014.bib, 
% you should add the following two lines:
\bibliographystyle{acl}
\bibliography{nodalida2015}

% Otherwise you can include your references as follows:
%% \begin{thebibliography}{}

%% \bibitem[\protect\citename{Aho and Ullman}1972]{Aho:72}
%% Alfred~V. Aho and Jeffrey~D. Ullman.
%% \newblock 1972.
%% \newblock {\em The Theory of Parsing, Translation and Compiling}, volume~1.
%% \newblock Prentice-{Hall}, Englewood Cliffs, NJ.

%% \bibitem[\protect\citename{{American Psychological Association}}1983]{APA:83}
%% {American Psychological Association}.
%% \newblock 1983.
%% \newblock {\em Publications Manual}.
%% \newblock American Psychological Association, Washington, DC.

%% \bibitem[\protect\citename{{Association for Computing Machinery}}1983]{ACM:83}
%% {Association for Computing Machinery}.
%% \newblock 1983.
%% \newblock {\em Computing Reviews}, 24(11):503--512.

%% \bibitem[\protect\citename{Chandra \bgroup et al.\egroup }1981]{Chandra:81}
%% Ashok~K. Chandra, Dexter~C. Kozen, and Larry~J. Stockmeyer.
%% \newblock 1981.
%% \newblock Alternation.
%% \newblock {\em Journal of the Association for Computing Machinery},
%%   28(1):114--133.

%% \bibitem[\protect\citename{Gusfield}1997]{Gusfield:97}
%% Dan Gusfield.
%% \newblock 1997.
%% \newblock {\em Algorithms on Strings, Trees and Sequences}.
%% \newblock Cambridge University Press, Cambridge, UK.

%% \end{thebibliography}

\end{document}

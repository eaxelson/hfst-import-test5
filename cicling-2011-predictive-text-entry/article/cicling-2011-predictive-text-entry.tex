\documentclass{llncs}
\usepackage{epsf}

\begin{document}

\title{Predictive Text Entry of Agglutinative Languages using Morphological Segmentation and Phonological Restrictions}

\author{\ldots}
\institute{\ldots}

\maketitle

\begin{abstract}

Linguistic models for predictive text entry on ambiguous keyboards
typically rely on large dictionaries, which are used to disambiguate
between words matching user input. This approach is insufficient for
heavily agglutinative languages, like Finnish or Turkish, where
morphological phenomena such as inflection and compounding increase
the rate of out-of-vocabulary words. We propose a method for text
entry, which circumvents the problem of out-of-vocabulary words, by
replacing the dictionary with a Markov chain on morpheme sequences
constructed from morphologically segmented training data. The Markov
chain is combined with a third order hidden Markov model (HMM) mapping
key sequences to letter sequences. Additionally we use rules, which
enforce phonotactic restrictions such as vowel harmony in Finnish. We
evaluate our method by constructing text entry systems for Finnish and
Turkish mobile phone keypads. We compare the Turkish text entry system
with an existing system, which is based on an HMM of letter sequences
\cite{Tantug:2010} and show that we achieve superior results measured
by the keystrokes per character ratio (KSPC)
\cite{MacKenzie02kspc}. We also compare the Finnish text entry system
to an existing system, which utilizes a dictionary
\cite{silfverberg/2011/cla} and show that we achieve superior
KSPC. For segmenting the training data, we use Morfessor, a system for
unsupervised morphological segmentation \cite{Creutz07ACMTSLP}. For
constructing the probabilistic models needed for the text entry
systems, we use tools for POS tagging from the HFST interface
\cite{Silfverberg/2011}, which is an open-source interface for
weighted finite-state calculus. We also utilize an open-source
two-level phonology rule compiler, hfst-twolc, for implementing the
vowel harmony rules needed for text entry of Finnish \cite{hfst/2011}.

\end{abstract}

\section{Introduction}

\section{Earlier Approaches to Predictive Text Entry}

\section{A Probabilistic Model of Word Structure}

\subsection{An Hidden Markov Model for Predicting Letter Sequences from Key Sequences}

\subsection{A Markov Chain of Morphs}

\subsection{Phonological Constraints}

\subsection{Combining Models using Weighted Finite-State Calculus}

\section{Training Materials and Test Materials for Finnish and Turkish}

\section{Evaluation}

\section{Discussion}

\section{Conclusions}

\section{Acknowledgments}

\bibliographystyle{splncs03}
\bibliography{cicling2011.bib}
\end{document}
